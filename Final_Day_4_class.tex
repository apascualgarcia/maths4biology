
%%%%%%%%%%%%%%%%%%%%%%%%%%%%%%%%%%%%%%%%%%%%%%%%%%%%%%%%%%%
% Actual text starts below:
%%%%%%%%%%%%%%%%%%%%%%%%%%%%%%%%%%%%%%%%%%%%%%%%%%%%%%%%%%%
%\setcounter{page}{25}

\section{Vector Algebra}
\subsection{Vectors}
Until now we have dealt only with  \emph{scalars} which are
one-dimensional entities consisting of a magnitude and a sign.
Higher-dimensional entities are composed of several scalars each
of which is related to a particular direction. These objects are called
vectors and are represented in print by either using bold symbols like
{\boldmath $x$} or with an arrow on top as in $\vec{x}$. An
$n$-dimensional vector has $n$ components $x_i$ with $i=1,...,n$.
Its magnitude is given by $\abs{\vec{x}}=\sqrt{x_1^2+x_2^2+...+x_n^2}$.

{\bf Notation:} $\quad \vec{x}=\left(\ba{c}x_1 \\ x_2 \\ \cdots \\x_n \ea \right)$
is a column vector and $\vec{x} = (x_1, x_2, ..., x_n)$ is row vector.

Sometimes a row vector is specifically denoted as $\vec{x}^T$ ($T$ for transposed).

A vector is graphically represented by an arrow. The vector's magnitude
$\abs{\vec{x}}$ is denoted by the arrow's length. If the starting point of the
vector coincides with the origin of the coordinate system, then
its end point corresponds to the coordinates of the vector components.
Such a vector is called a coordinate vector. \vs

\begin{figure}[!h]
    \centerline{\epsfxsize=8cm  \epsfbox{matlab/fig24.eps}} \svs
    \caption{The vector $(1,2)$ is an arrow from the origin to the point $x=1$ and $y=2$.} \label{fig27}
\end{figure}


\subsection{Elementary Vector Operations}

\subsubsection{Addition and Subtraction}

The sum two vectors can be obtained graphically by either shifting
the tail of the second arrow to the head of the first, or by constructing
the parallelogram that is defined by the two arrows. The difference between
two vectors can be found by adding the vector that has the same length but
points into the opposite direction.

\begin{figure}[!h]
\centering \subfigure[The sum of two vectors
$\vec{c}=\vec{a}+\vec{b}$]{\epsfxsize=6cm
\epsfbox{matlab/fig25a.eps}} \hspace*{1cm} \subfigure[The
difference between two vectors $\vec{c}=\vec{a}-\vec{b}$]
           {\epsfxsize=6cm \epsfbox{matlab/fig25b.eps}} \svs
\caption{Addition and subtraction of vectors} \label{fig28}
\end{figure}

In components:  $\quad \vec{a} + \vec{b}=(a_1+b_1, ...,a_n+b_n)=(c_1, ...,c_n) = \vec{c}$ \vs

{\bf Properties:}
\bnn \begin{array}{cccc}
    \vec{a} + \vec{b} & = & \vec{b} + \vec{a} &  \qquad\quad \mbox{\bf commutative} \qquad\quad \\
    (\vec{a} + \vec{b}) + \vec{c} & = & \vec{a} + (\vec{b} +\vec{c}) & \qquad\quad \mbox{\bf associative} \qquad\quad
\end{array} \enn \vs

A closed polygon corresponds to the vector sum equal $\vec{0}$. \svs

\begin{figure}[!h]
    \centerline{\epsfxsize=6cm  \epsfbox{matlab/fig26.eps}} \svs
    \caption{$\vec{a_1}+\vec{a_2}+\ldots + \vec{a_5}=\vec{0}$} \label{fig30}
\end{figure}

\vs {\bf Important note:} Make sure you understand that $\vec{0} \neq 0$ !!!!

\subsubsection{Multiplication of a Vector with a Scalar}
A vector can be multiplied with a scalar by multiplying each of the components
which results in either stretching or squeezing  of the vector and may change
its orientation.
\bnn
\vec{a}=\left( \begin{array}{c} 1 \\ 1 \end{array} \right) \quad \qquad
\vec{b}=-2 \, \vec{a}=-2 \left( \begin{array}{c} 1 \\ 1 \end{array} \right)
   = \left( \begin{array}{c} -2 \\ -2\\ \end{array} \right)
\enn \svs

\begin{figure}[!h]
    \centerline{\epsfxsize=6cm \epsfbox{matlab/fig27.eps}} \svs
    \caption{The multiplication of a vector with a scalar $\vec{b}=-2\,\vec{a}$} \label{fig31}
\end{figure} \vs

\textbf{Linear dependence of vectors:}

$n$ vectors $\vec{a_1}, \cdots, \vec{a_n}$ are called {\em linearly independent}, if
the only way to fulfill
\bnn
    \alpha_1 \, \vec{a_1} + \alpha_2 \, \vec{a_2} + \cdots + \alpha_n \, \vec{a_n}=0
    \quad \mbox{is} \quad \alpha_1=\alpha_2=\cdots=\alpha_n=0
\enn

If this relation can be fulfilled with certain $\alpha_i \neq 0$, then the vectors
are said to be {\em linearly dependent}. For instance, imagine $\alpha_1 \not = 0$,
all others are free. Then $\vec{a_1}$ may be expressed by the other vectors and is
redundant.
\beq \vec{a_1}= -\frac{1}{\alpha_1} \sum_{i=2}^n \alpha_i \, \vec{a_i} \eeq

One-dimensional: $\; \; \alpha \, \vec{a}$ represents all vectors on a straight line.
Such vectors are called collinear.

Two-dimensional: $\;\; \alpha_1 \, \vec{a_1} +\alpha_2 \, \vec{a_2}$
represents all vectors in the plane. These vectors are coplanar.
\begin{figure}[!h]
    \centering
    \subfigure[Collinear vectors define a line]{\epsfxsize=4cm  \epsfbox{matlab/fig28a.eps}}
    \hspace*{1cm}
    \subfigure[Two non-collinear vectors span a plane]{\epsfxsize=7cm  \epsfbox{matlab/fig28b.eps}} \svs
    \caption{Collinear and coplanar vectors} \label{fig33}
\end{figure}

\subsubsection{Scalar Product}

Two vectors $\vec{a}$ and $\vec{b}$ can be multiplied such that the result is a scalar $c$.
This operation is called the {\em scalar, dot} or {\em inner} product.
\bnn \begin{array}{cc} \svs
    \vec{a}\cdot\vec{b}=\abs{\vec{a}} \; \abs{\vec{b}} \cos \alpha & \qquad
    \mbox{where $\alpha$ is the angle between $\vec{a}$ and $\vec{b}$} \qquad \\
    \vec{a}\cdot\vec{b}=a_{1} \, b_{1}+a_{2} \, b_{2}+...+ a_{n} \, b_{n} & \qquad
    \mbox{scalar product in components} \qquad
\end{array} \enn

The scalar product measures the contribution of vector $\vec{a}$ to vector $\vec{b}$
If the angle between $\vec{a}$ and $\vec{b}$ is $90^{\circ}$ the two vectors are orthogonal, there are no contributions at all. \vs

{\bf Properties:}
\bnn \begin{array}{cc} \svs
    \vec{a}\cdot\vec{b}=\vec{b}\cdot\vec{a} & \qquad \mbox{\bf commutative} \qquad \\ \svs
    (c\, \vec{a})\cdot \vec{b}=c \, (\vec{a}\cdot\vec{b})=\vec{a}\cdot\,(c \, \vec{b}) & \qquad \mbox{\bf associative} \qquad \\
    (\vec{a_1}+\vec{a_2})\cdot\vec{b}=\vec{a_1}\cdot\vec{b}+\vec{a_2}\cdot\vec{b} & \qquad \mbox{\bf distributive} \qquad
\end{array} \enn

{\bf Examples:}
\bnn
\vec{a}= \left( \begin{array}{c} 2 \\ 0  \end{array} \right) \qquad
\vec{b}= \left( \begin{array}{c}  1 \\  -2 \end{array} \right) \qquad
\vec{a}\cdot\vec{b}=2\cdot1+0\cdot(-2)=2
\enn
\bnn
\qquad\qquad \abs{\vec{a}}= 2 \qquad \abs{\vec{b}}=\sqrt{5}
\quad \rightarrow \quad \cos\alpha=\frac{\vec{a} \cdot \vec{b}}{\abs{\vec{a}}\,\abs{\vec{b}}}
=\frac{2}{2 \sqrt{5}} \quad \rightarrow \quad \alpha =\arccos\frac{1}{\sqrt{5}} = 1.107 \approx 63^{\circ}
\enn

\begin{figure}[!ht]
    \centering
    \subfigure[Projection of $\vec{b}$ on $\vec{a}$]{\epsfxsize=6cm  \epsfbox{matlab/fig29a.eps}}
    \hspace*{1cm}
    \subfigure[If $\alpha>90^0$ then $\cos \alpha < 0$ and the scalar
        product is negative]{\epsfxsize=6cm \epsfbox{matlab/fig29b.eps}} \svs
    \caption{Scalar Product}  \label{fig35}
\end{figure} \vs

\begin{figure}[!h]
    \centering
    \subfigure[The dot product has is maximal value in the case $\alpha = 0 \; \rightarrow \cos \alpha =1$]
    {\epsfxsize=6cm  \epsfbox{matlab/fig30a.eps}}
    \hspace*{1cm}
    \subfigure[For $\alpha = 90^{\circ}$ the scalar product vanishes because $\cos \alpha =0$]
    {\epsfxsize=6cm  \epsfbox{matlab/fig30b.eps}} \svs
    \caption{Scalar product for parallel and orthogonal vectors}
\end{figure}

\newpage

\subsubsection{Vector Product}

Two vectors $\vec{a}$ and $\vec{b}$ can be multiplied such that the result is a vector $\vec{c}$.
This operation is called the {\em vector, cross} or {\em outer} product.

\vs \centerline{\bf The vector product exists only in three dimensions !!!}
\bnn
\vec{a}\times \vec{b} = \vec{c} \qquad\qquad
\abs{\vec{c}} = \abs{\vec{a} \times \vec{b}} = \abs{\vec{a}} \, \abs{\vec{b}} \, \sin \alpha
\enn

\bnn
\left(\begin{array}{c} a_{1} \\ a_{2} \\ a_{3} \\ \end{array}\right) \times
\left(\begin{array}{c} b_{1} \\ b_{2} \\ b_{3} \\ \end{array}\right)
=\left(\begin{array}{c} a_{2}b_{3}-a_{3}b_{2} \\ a_{3}b_{1}-a_{1}b_{3} \\ a_{1}b_{2}-a_{2}b_{1} \end{array}\right)
\qquad \mbox{vector product in components}
\enn

The result of a vector product between two non-collinear vectors $\vec{a}$ and $\vec{b}$ is a vector $\vec{c}$ 
which has a magnitude of $\abs{\vec{a}}\,\abs{\vec{b}} \, \sin\alpha$ and points into the direction perpendicular
to the plane defined by $\vec{a}$ and $\vec{b}$ such that $\vec{a}$, $\vec{b}$ and $\vec{c}$ form a right-handed 
system. To find this direction have your right thumb point into the direction of $\vec{a}$, the right index into
the direction of $\vec{b}$, and the right middle finger perpendicular to the plane defined by $\vec{a}$ and $\vec{b}$. 
There is only one way to do that without hurting yourself seriously. Now the middle finger points into the direction 
of $\vec{c}$. 

Hint: It is imperative that you use the {\bf right} hand for this. \svs

{\bf Properties:}
\bnn \begin{array}{cc} \svs
    \vec{a} \times \vec{b}=-\vec{b} \times \vec{a} & \qquad \qquad \mbox{anti-commutative} \\ \svs
    (c\,\vec{a}) \times \vec{b}= c\,(\vec{a} \times \vec{b})= \vec{a} \times (c\, \vec{b}) & \qquad\qquad \mbox{associative} \\
    (\vec{a_1}+\vec{a_2}) \times \vec{b}=\vec{a_1} \times \vec{b}+\vec{a_2} \times \vec{b} & \qquad\qquad \mbox{distributive}
\end{array} \enn \svs

{\bf Note:} In 3 dimensions a plane can be defined by a point in the plane and its {\em normal vector} $\vec{n}$. \vs

\begin{figure}[!h]
    \centering
    \subfigure[$\vec{a}$ and $\vec{b}$ span a plane]{\epsfxsize=7cm  \epsfbox{matlab/fig31a.eps}}
    \hspace{0.5cm}
    \subfigure[A plane defined by a point and its normal vector]{\epsfxsize=7cm  \epsfbox{matlab/fig31b.eps}} \svs
    \caption{Vectors in 3-dimensional space}  \label{fig39}
\end{figure} \vs


\subsection{Basis vectors}
Basis vectors span a coordinate system and can be represented in various ways
\bnn \vec{i}, \vec{j}, \vec{k} \qquad \vec{e_1},\vec{e_2},\vec{e_3} \qquad
\left(\begin{array}{c} 1 \\ 0 \\ 0 \\ \end{array}\right),
\left(\begin{array}{c} 0 \\ 1 \\ 0 \\ \end{array}\right),
\left(\begin{array}{c} 0 \\ 0 \\ 1 \\ \end{array}\right) \enn

\bnn \vec{s}= \left(\begin{array}{c} x \\ y \\ \end{array} \right)
 = x \, \overbrace{\left(\begin{array}{c} 1 \\ 0 \\ \end{array} \right)}^{\overrightarrow{e_1}}
 +  y \, \overbrace{\left(\begin{array}{c} 0 \\ 1 \\ \end{array}\right)}^{\overrightarrow{e_2}}
 = x \, \vec{e_1}+y \, \vec{e_2} \enn

\vs \begin{figure}[!ht]
    \centerline{\epsfxsize=6cm  \epsfbox{matlab/fig33.eps}}
    \caption{Basis vectors.} \label{fig42}
\end{figure}



