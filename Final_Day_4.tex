
%%%%%%%%%%%%%%%%%%%%%%%%%%%%%%%%%%%%%%%%%%%%%%%%%%%%%%%%%%%
% Actual text starts below:
%%%%%%%%%%%%%%%%%%%%%%%%%%%%%%%%%%%%%%%%%%%%%%%%%%%%%%%%%%%
%\setcounter{page}{25}

\section{Algebra}
\subsection{Vectors}
Until now we have dealt only with  \emph{scalars} which are
one-dimensional entities consisting of a magnitude and a sign.
Higher-dimensional entities are composed of several scalars each
of which is related to a particular direction. These objects are called
vectors and are represented in print by either using bold symbols like
{\boldmath $x$} or with an arrow on top as in $\vec{x}$. An
$n$-dimensional vector has $n$ components $x_i$ with $i=1,...,n$.
Its magnitude is given by $\abs{\vec{x}}=\sqrt{x_1^2+x_2^2+...+x_n^2}$.

{\bf Notation:} $\quad \vec{x}=\left(\ba{c}x_1 \\ x_2 \\ \cdots \\x_n \ea \right)$
is a column vector and $\vec{x} = (x_1, x_2, ..., x_n)$ is row vector.

Sometimes a row vector is specifically denoted as $\vec{x}^T$ ($T$ for transposed).

A vector is graphically represented by an arrow. The vector's magnitude
$\abs{\vec{x}}$ is denoted by the arrow's length. If the starting point of the
vector coincides with the origin of the coordinate system, then
its end point corresponds to the coordinates of the vector components.
Such a vector is called a coordinate vector. \vs

\begin{figure}[!h]
    \centerline{\epsfxsize=10cm  \epsfbox{matlab/fig24.eps}} \svs
    \caption{The vector $(1,2)$ is an arrow from the origin to the point $x=1$ and $y=2$.} \label{fig27}
\end{figure}


\subsection{Elementary Vector Operations}

\subsubsection{Addition and Subtraction}

The sum two vectors can be obtained graphically by either shifting
the tail of the second arrow to the head of the first, or by constructing
the parallelogram that is defined by the two arrows. The difference between
two vectors can be found by adding the vector that has the same length but
points into the opposite direction.

\begin{figure}[!h]
\centering \subfigure[The sum of two vectors
$\vec{c}=\vec{a}+\vec{b}$]{\epsfxsize=7cm
\epsfbox{matlab/fig25a.eps}} \hspace*{0.5cm} \subfigure[The
difference between two vectors $\vec{c}=\vec{a}-\vec{b}$]
           {\epsfxsize=7.5cm \epsfbox{matlab/fig25b.eps}} \svs
\caption{Addition and subtraction of vectors} \label{fig28}
\end{figure}

In components:  $\quad \vec{a} + \vec{b}=(a_1+b_1, ...,a_n+b_n)=(c_1, ...,c_n) = \vec{c}$ \vs

{\bf Properties:}
\bnn \begin{array}{cccc}
    \vec{a} + \vec{b} & = & \vec{b} + \vec{a} &  \qquad\quad \mbox{\bf commutative} \qquad\quad \\
    (\vec{a} + \vec{b}) + \vec{c} & = & \vec{a} + (\vec{b} +\vec{c}) & \qquad\quad \mbox{\bf associative} \qquad\quad
\end{array} \enn \vs

A closed polygon corresponds to the vector sum equal $\vec{0}$. \svs

\begin{figure}[!h]
    \centerline{\epsfxsize=8cm  \epsfbox{matlab/fig26.eps}} \svs
    \caption{$\vec{a_1}+\vec{a_2}+\ldots + \vec{a_5}=\vec{0}$} \label{fig30}
\end{figure}

\vs {\bf Important note:} Make sure you understand that $\vec{0} \neq 0$ !!!!

\subsubsection{Multiplication of a Vector with a Scalar}
A vector can be multiplied with a scalar by multiplying each of the components
which results in either stretching or squeezing  of the vector and may change
its orientation.
\bnn
\vec{a}=\left( \begin{array}{c} 1 \\ 1 \end{array} \right) \quad \qquad
\vec{b}=-2 \, \vec{a}=-2 \left( \begin{array}{c} 1 \\ 1 \end{array} \right)
   = \left( \begin{array}{c} -2 \\ -2\\ \end{array} \right)
\enn \svs

\begin{figure}[!h]
    \centerline{\epsfxsize=8cm \epsfbox{matlab/fig27.eps}} \svs
    \caption{The multiplication of a vector with a scalar $\vec{b}=-2\,\vec{a}$} \label{fig31}
\end{figure} \vs

\textbf{Linear dependence of vectors:}

$n$ vectors $\vec{a_1}, \cdots, \vec{a_n}$ are called {\em linearly independent}, if
the only way to fulfill
\bnn
    \alpha_1 \, \vec{a_1} + \alpha_2 \, \vec{a_2} + \cdots + \alpha_n \, \vec{a_n}=0
    \quad \mbox{is} \quad \alpha_1=\alpha_2=\cdots=\alpha_n=0
\enn

If this relation can be fulfilled with certain $\alpha_i \neq 0$, then the vectors
are said to be {\em linearly dependent}. For instance, imagine $\alpha_1 \not = 0$,
all others are free. Then $\vec{a_1}$ may be expressed by the other vectors and is
redundant.
\beq \vec{a_1}= -\frac{1}{\alpha_1} \sum_{i=2}^n \alpha_i \, \vec{a_i} \eeq

One-dimensional: $\; \; \alpha \, \vec{a}$ represents all vectors on a straight line.
Such vectors are called collinear.

Two-dimensional: $\;\; \alpha_1 \, \vec{a_1} +\alpha_2 \, \vec{a_2}$
represents all vectors in the plane. These vectors are coplanar.
\begin{figure}[!h]
    \subfigure[Collinear vectors define a line]{\epsfxsize=5cm  \epsfbox{matlab/fig28a.eps}}
    \hspace*{0.5cm}
    \subfigure[Two non-collinear vectors span a plane]{\epsfxsize=9cm  \epsfbox{matlab/fig28b.eps}} \svs
    \caption{Collinear and coplanar vectors} \label{fig33}
\end{figure}

\subsubsection{Scalar Product}

Two vectors $\vec{a}$ and $\vec{b}$ can be multiplied such that the result is a scalar $c$.
This operation is called the {\em scalar, dot} or {\em inner} product.
\bnn \begin{array}{cc} \svs
    \vec{a}\cdot\vec{b}=\abs{\vec{a}} \; \abs{\vec{b}} \cos \alpha & \qquad
    \mbox{where $\alpha$ is the angle between $\vec{a}$ and $\vec{b}$} \qquad \\
    \vec{a}\cdot\vec{b}=a_{1} \, b_{1}+a_{2} \, b_{2}+...+ a_{n} \, b_{n} & \qquad
    \mbox{scalar product in components} \qquad
\end{array} \enn

The scalar product measures the contribution of vector $\vec{a}$ to vector $\vec{b}$
If the angle between $\vec{a}$ and $\vec{b}$ is $90^{\circ}$ the two vectors are orthogonal, there are no contributions at all. \vs

{\bf Properties:}
\bnn \begin{array}{cc} \svs
    \vec{a}\cdot\vec{b}=\vec{b}\cdot\vec{a} & \qquad \mbox{\bf commutative} \qquad \\ \svs
    (c\, \vec{a})\cdot \vec{b}=c \, (\vec{a}\cdot\vec{b})=\vec{a}\cdot\,(c \, \vec{b}) & \qquad \mbox{\bf associative} \qquad \\
    (\vec{a_1}+\vec{a_2})\cdot\vec{b}=\vec{a_1}\cdot\vec{b}+\vec{a_2}\cdot\vec{b} & \qquad \mbox{\bf distributive} \qquad
\end{array} \enn

{\bf Examples:}
\bnn
\vec{a}= \left( \begin{array}{c} 2 \\ 0  \end{array} \right) \qquad
\vec{b}= \left( \begin{array}{c}  1 \\  -2 \end{array} \right) \qquad
\vec{a}\cdot\vec{b}=2\cdot1+0\cdot(-2)=2
\enn
\bnn
\qquad\qquad \abs{\vec{a}}= 2 \qquad \abs{\vec{b}}=\sqrt{5}
\quad \rightarrow \quad \cos\alpha=\frac{\vec{a} \cdot \vec{b}}{\abs{\vec{a}}\,\abs{\vec{b}}}
=\frac{2}{2 \sqrt{5}} \quad \rightarrow \quad \alpha =\arccos\frac{1}{\sqrt{5}} = 1.107 \approx 63^{\circ}
\enn

\begin{figure}[!ht]
    \centering
    \subfigure[Projection of $\vec{b}$ on $\vec{a}$]{\epsfxsize=7cm  \epsfbox{matlab/fig29a.eps}}
    \hspace*{0.5cm}
    \subfigure[If $\alpha>90^0$ then $\cos \alpha < 0$ and the scalar
        product is negative]{\epsfxsize=7cm \epsfbox{matlab/fig29b.eps}} \svs
    \caption{Scalar Product}  \label{fig35}
\end{figure} \vs

\begin{figure}[!h]
    \centering
    \subfigure[The dot product has is maximal value in the case $\alpha = 0 \; \rightarrow \cos \alpha =1$]
    {\epsfxsize=7cm  \epsfbox{matlab/fig30a.eps}}
    \hspace*{0.5cm}
    \subfigure[For $\alpha = 90^{\circ}$ the scalar product vanishes because $\cos \alpha =0$]
    {\epsfxsize=7cm  \epsfbox{matlab/fig30b.eps}} \svs
    \caption{Scalar product for parallel and orthogonal vectors}
\end{figure}

\newpage

\subsubsection{Vector Product}

Two vectors $\vec{a}$ and $\vec{b}$ can be multiplied such that the result is a vector $\vec{c}$.
This operation is called the {\em vector, cross} or {\em outer} product.

\vs \centerline{\bf The vector product exists only in three dimensions !!!}
\bnn
\vec{a}\times \vec{b} = \vec{c} \qquad\qquad
\abs{\vec{c}} = \abs{\vec{a} \times \vec{b}} = \abs{\vec{a}} \, \abs{\vec{b}} \, \sin \alpha
\enn

\bnn
\left(\begin{array}{c} a_{1} \\ a_{2} \\ a_{3} \\ \end{array}\right) \times
\left(\begin{array}{c} b_{1} \\ b_{2} \\ b_{3} \\ \end{array}\right)
=\left(\begin{array}{c} a_{2}b_{3}-a_{3}b_{2} \\ a_{3}b_{1}-a_{1}b_{3} \\ a_{1}b_{2}-a_{2}b_{1} \end{array}\right)
\qquad \mbox{vector product in components}
\enn

The result of a vector product between two non-collinear vectors $\vec{a}$ and $\vec{b}$ is a vector $\vec{c}$ 
which has a magnitude of $\abs{\vec{a}}\,\abs{\vec{b}} \, \sin\alpha$ and points into the direction perpendicular
to the plane defined by $\vec{a}$ and $\vec{b}$ such that $\vec{a}$, $\vec{b}$ and $\vec{c}$ form a right-handed 
system. To find this direction have your right thumb point into the direction of $\vec{a}$, the right index into
the direction of $\vec{b}$, and the right middle finger perpendicular to the plane defined by $\vec{a}$ and $\vec{b}$. 
There is only one way to do that without hurting yourself seriously. Now the middle finger points into the direction 
of $\vec{c}$. 

Hint: It is imperative that you use the {\bf right} hand for this. \svs

{\bf Properties:}
\bnn \begin{array}{cc} \svs
    \vec{a} \times \vec{b}=-\vec{b} \times \vec{a} & \qquad \qquad \mbox{anti-commutative} \\ \svs
    (c\,\vec{a}) \times \vec{b}= c\,(\vec{a} \times \vec{b})= \vec{a} \times (c\, \vec{b}) & \qquad\qquad \mbox{associative} \\
    (\vec{a_1}+\vec{a_2}) \times \vec{b}=\vec{a_1} \times \vec{b}+\vec{a_2} \times \vec{b} & \qquad\qquad \mbox{distributive}
\end{array} \enn \svs

{\bf Note:} In 3 dimensions a plane can be defined by a point in the plane and its {\em normal vector} $\vec{n}$. \vs

\begin{figure}[!h]
    \centering
    \subfigure[$\vec{a}$ and $\vec{b}$ span a plane]{\epsfxsize=7cm  \epsfbox{matlab/fig31a.eps}}
    \hspace{0.5cm}
    \subfigure[A plane defined by a point and its normal vector]{\epsfxsize=7cm  \epsfbox{matlab/fig31b.eps}} \svs
    \caption{Vectors in 3-dimensional space}  \label{fig39}
\end{figure} \vs

\subsection{Matrices}
A matrix {\bf A} operates on a vector $\vec{x}$ and transforms, i.e. stretches, squeezes or rotates it.

\bnn
\vec{y}= \mbox{\boldmath $A$} \, \vec{x} \qquad \mbox{where} \quad
\mbox{\boldmath $A$}=\left(\begin{array}{lll} A_{11} & A_{12} & A_{1n} \\ A_{21} & \ddots& \ldots \\
A_{n1} & \cdots &  A_{nn} \end{array} \right)=A_{ij} \quad \mbox{is a $\; n \times n$ matrix}
\enn \svs

\bnn
\vec{y}=\left( \begin{array}{c} y_{1} \\ y_{2} \\ \vdots \\ y_{n}  \\ \end{array}\right) \;=\;
\left(\begin{array}{c} A_{11} \, x_{1}+A_{12} \, x_{2}+\ldots A_{1n} \, x_{n}  \\
 A_{21} \, x_{1}+A_{22} \, x_{2}+\ldots A_{2n} \, x_{n} \\ \vdots \\
 A_{n1} \, x_{1}+A_{n2} \, x_{2}+\ldots A_{nn} \, x_{n} \\ \end{array} \right)
 \enn \svs

\bnn
\underbrace{\left( \begin{array}{cr} 1 & 1 \\ 1 & -2 \\ \end{array} \right)}_{\mat{A}} \;
\underbrace{ \left( \begin{array}{c} 1 \\ 1 \\ \end{array} \right)}_{\vec{x}}
\; = \; \underbrace{\left( \begin{array}{r} 2\\ -1 \\ \end{array} \right)}_{\vec{y}}
\enn

\begin{figure}[!h]
    \centerline{\epsfxsize=9cm  \epsfbox{matlab/fig32.eps}}
    \caption{Rotation and scaling of a vector}  \label{fig41}
\end{figure} \vs

{\bf Properties:}
\bnn \mat{A} = \mat{B} \quad \rightarrow \quad a_{ij}=b_{ij}  \enn
\bnn \mat{A} + \mat{B} \quad \rightarrow \quad a_{ij}+b_{ij}  \enn
\bnn c \, \mat{A} \quad \rightarrow \quad c \, a_{ij}  \enn
\bnn \mat{A}+\mat{B}  =   \mat{B} + \mat{A}  \enn

\subsection{Multiplication of Matrices}
The product of two matrices $\mat{A}$ and $\mat{B}$ is found by calculating
the scalar products between the {\it rows} of matrix $\mat{A}$ and the 
{\it columns} of matrix $\mat{B}$. This implies that the number of columns
of matrix $\mat{A}$ must be the same as the number of rows of matrix $\mat{B}$. \svs

{\bf Examples:}
\bnn 
    \left( \begin{array}{cc} a_{11} & a_{12} \\ a_{21} & a_{22} \end{array} \right)  
    \left( \begin{array}{cc} b_{11} & b_{12} \\ b_{21} & b_{22} \end{array} \right) \,=\, 
    \left( \begin{array}{cc} a_{11}\,b_{11}+a_{12}\,b_{21} & a_{11}\,b_{12}+a_{12}\,b_{22} \\ 
                         a_{21}\,b_{11}+a_{22}\,b_{21} & a_{21}\,b_{12}+a_{22}\,b_{22} \end{array} \right) 
\enn

\bnn
    \mat{A}\,\mat{B} = 
    \left( \begin{array}{cc} 5 & 3 \\ 2 & 7 \end{array} \right)  
    \left( \begin{array}{cc} 2 & -3 \\ -1 & 4 \end{array} \right) \,=\, 
    \left( \begin{array}{cc} 5\cdot2+3\cdot(-1) & 5\cdot (-3)+3\cdot 4 \\ 
                         2\cdot 2+7\cdot (-1) & 2 \cdot (-3)+7\cdot 4 \end{array} \right) \,=\,
    \left( \begin{array}{cc} 7 & -3 \\  -3 & 22 \end{array} \right) 
\enn
\bnn
    \mat{B}\,\mat{A} = 
    \left( \begin{array}{cc} 2 & -3 \\ -1 & 4 \end{array} \right)  
    \left( \begin{array}{cc} 5 & 3 \\ 2 & 7 \end{array} \right) \,=\,
    \left( \begin{array}{cc} 4 & -15 \\  3 & -25 \end{array} \right) \,\neq \, \mat{A} \, \mat{B} 
\enn \vs

\centerline{\bf The multiplication of matrices is NOT commutative!!!!!}

In general a $n \times m$ matrix can be multiplied with a $m \times n$ matrix and the result is a
$n \times n$ matrix.

\bnn  \mat{C} = \mat{A} \, \mat{B} = \left(\begin{array}{cccc}
\Sigma_i \: a_{1i} \, b_{i1} & \Sigma_i \, a_{1i} \: b_{i2} & \hdots & \Sigma_i \, a_{1i} \, b_{in}\\
\Sigma_i \: a_{2i} \, b_{i1} & \Sigma_i \, a_{2i} \: b_{i2} & \hdots & \Sigma_i \, a_{2i} \, b_{in}\\
\vdots & \vdots & & \vdots \\
\Sigma_i \: a_{ni} \, b_{i1} & \Sigma_i \, a_{ni} \: b_{i2} & \hdots & \Sigma_i \, a_{ni} \, b_{in}
\end{array}\right) \qquad \mbox{with} \;\; \Sigma_i = \sum_{i=1}^m \enn \svs

\subsection{Transposed Matrix}
The transposed of a matrix is found by exchanging the row and column vectors.

\bnn 
\mat{A}=\left(\begin{array}{cccc} A_{11} & A_{12} & A_{13} & \ldots \\
  A_{21} & A_{22} & \cdot & \cdots \\ \vdots & \vdots & \ddots & \vdots\\
  \cdot & \cdot& \cdots & A_{mn} \end{array}\right) \quad \rightarrow \quad
\mat{A^T}=\left(\begin{array}{cccc} A_{11} & A_{21} & A_{31} & \ldots \\
  A_{12} & A_{22} & \cdot & \cdots \\ \vdots & \vdots & \ddots & \vdots\\
  \cdot & \cdot& \cdots & A_{nm} \end{array}\right) 
\enn \svs

{\bf Examples:}
\bnn
    \mat{A} = \left( \begin{array}{ccc} 2 & 3 & 5 \\ 1 & 4 & 0 \end{array} \right) \quad \rightarrow \quad 
    \mat{A^T} = \left( \begin{array}{cc} 2 & 1 \\ 3 & 4 \\ 5 & 0 \end{array} \right) 
\enn

\bnn
    \mat{A}\,\mat{A^T} = 
    \left( \begin{array}{ccc} 2 & 3 & 5 \\ 1 & 4 & 0 \end{array} \right)  
    \left( \begin{array}{cc} 2 & 1 \\ 3 & 4 \\ 5 & 0 \end{array} \right) = 
    \left( \begin{array}{cc} 38 & 14 \\ 14 & 17 \end{array} \right)
\enn

\bnn
    \mat{A^T}\,\mat{A} = 
    \left( \begin{array}{cc} 2 & 1 \\ 3 & 4 \\ 5 & 0 \end{array} \right) 
    \left( \begin{array}{ccc} 2 & 3 & 5 \\ 1 & 4 & 0 \end{array} \right) =  
    \left( \begin{array}{ccc} 5 & 10 & 10 \\ 10 & 25 & 15 \\ 10 & 15 & 25 \end{array} \right)
\enn

This should eliminate any remaining doubts that matrix multiplication could be commutative.
These matrices are not even the same size.

\subsection{Basis vectors}
Basis vectors span a coordinate system and can be represented in various ways
\bnn \vec{i}, \vec{j}, \vec{k} \qquad \vec{e_1},\vec{e_2},\vec{e_3} \qquad
\left(\begin{array}{c} 1 \\ 0 \\ 0 \\ \end{array}\right),
\left(\begin{array}{c} 0 \\ 1 \\ 0 \\ \end{array}\right),
\left(\begin{array}{c} 0 \\ 0 \\ 1 \\ \end{array}\right) \enn

\bnn \vec{s}= \left(\begin{array}{c} x \\ y \\ \end{array} \right)
 = x \, \overbrace{\left(\begin{array}{c} 1 \\ 0 \\ \end{array} \right)}^{\overrightarrow{e_1}}
 +  y \, \overbrace{\left(\begin{array}{c} 0 \\ 1 \\ \end{array}\right)}^{\overrightarrow{e_2}}
 = x \, \vec{e_1}+y \, \vec{e_2} \enn

\vs \begin{figure}[!ht]
    \centerline{\epsfxsize=7cm  \epsfbox{matlab/fig33.eps}}
    \caption{Basis vectors.} \label{fig42}
\end{figure}

\subsection{Transformation of coordinate systems}
In general the component of a vector depend on the coordinate system used. Coordinate
systems can be transformed, which changes the vector components in a certain way and a vector $\vec{s}$
in the old coordinate system becomes the vector $\widetilde{\vec{s}}$ in the new coordinates. The two easiest
transformations of a coordinate system are a translation or shift and a rotation around the origin.

\subsubsection{Translation}
A translation of the coordinate system is performed by adding a constant vector
\bnn 
\vec{s} \rightarrow \widetilde{\vec{s}}=\vec{s}+\vec{t} \qquad\mbox{shifts the coordinate system by} \;\;
\vec{t}= \left(\begin{array}{c} t_1 \\ t_2 \end{array}\right) 
\enn

Components of $\vec{s}$ in old system: $\vec{s}
=\left(\begin{array}{c} x+t_1 \\ y+t_2 \\ \end{array} \right)
=\left(\begin{array}{c} 1+t_1 \\ 2+t_2 \\ \end{array} \right)$

\svs
Components of $\widetilde{\vec{s}}$ in the new system: $\widetilde{\vec{s}}
=\left(\begin{array}{c} \widetilde{x} \\ \widetilde{y} \\ \end{array} \right)
=\left(\begin{array}{c} 2 \\ 1 \\ \end{array} \right)$

\begin{figure}[!h]
    \centering
    \subfigure[Translation of the coordinate system]{\epsfxsize=7cm  \epsfbox{matlab/fig34.eps}}
    \hspace{0.5cm}
    \subfigure[Rotation of the coordinate system]{\epsfxsize=7cm \epsfbox{matlab/fig35.eps}} \svs
    \caption{Coordinate transformations: Translation and Rotation} \label{fig43}
\end{figure}

\subsubsection{Rotation}
A rotation of the coordinate system by an angle $\alpha$ around the origin is performed by 
applying the rotation matrix $\mat{R}$ to the vector $\widetilde{\vec{s}}$
\bnn \mat{R} =  
    \left( \begin{array}{cc} \cos \alpha & \sin \alpha \\ -\sin \alpha & \cos \alpha \end{array} \right) 
    \qquad 
    \widetilde{\vec{s}} = \mat{R} \,\vec{s} = 
    \left( \begin{array}{cc} \cos \alpha & -\sin \alpha \\ \sin \alpha & \cos \alpha \end{array} \right) 
    \left( \begin{array}{c} x \\ y \end{array} \right) =
    \left( \begin{array}{c} x\,\cos \alpha - y\,\sin \alpha \\ x\,\sin \alpha + y\,\cos \alpha \end{array} \right) 
\enn

The rotation matrix $\mat{R}$ can be found by calculating the basis vectors $\widetilde{\vec{e_1}}$ 
and $\widetilde{\vec{e_2}}$ for the new coordinate system 
\bnn 
\widetilde{\vec{e_1}}= \cos\alpha \, \vec{e_{1}} + \sin\alpha \, \vec{e_2}  \qquad \mbox{and} \qquad
\widetilde{\vec{e_2}}=-\sin\alpha \, \vec{e_1} + \cos\alpha \, \vec{e_2} 
\enn

Representation of a point $\vec{s}$:
\bnn 
\vec{s} = \left( \begin{array}{c} x \\ y \end{array} \right) 
= \underbrace{x \, \vec{e_1} + y \, \vec{e_2}}_{\mbox{old system}}
= \underbrace{\widetilde{x} \, \widetilde{\vec{e_1}} + \widetilde{y} \, \widetilde{\vec{e_2}}}_{\mbox{new system}} 
= \left( \begin{array}{c} \widetilde{x}\,\cos \alpha + \widetilde{y}\,\sin \alpha \\ 
        -\widetilde{x}\,\sin \alpha + \widetilde{y}\,\cos \alpha \end{array} \right) 
\enn

Relation between old and new coordinates:
\bnn \begin{array}{cccc} \svs
    \vec{x} = \mat{A} \, \widetilde{\vec{x}} & \quad \mbox{where} \quad &
    \mat{A}=\left( \begin{array}{cc} \cos\alpha & -\sin\alpha \\ \sin\alpha &  \cos\alpha \end{array} \right) & \\
    \widetilde{\vec{x}}= \mat{A^{T}} \, \vec{x} & \quad \mbox{with} \quad &
    \mat{A^T}=\left( \begin{array}{cc} \cos\alpha & \sin\alpha \\ -\sin\alpha &  \cos\alpha \end{array} \right)
    & \quad \mbox{and}\quad \mat{A} \, \mat{A^{T}}=\left( \begin{array}{cc} 1 & 0 \\ 0 & 1 \end{array} \right)
\end{array} \enn

\bnn \begin{array}{ccc} \svs
    \vec{s}=x \, \vec{e_1} + y \, \vec{e_2} & = &
    \widetilde{x} \, \underbrace{(\cos\alpha \, \vec{e_1} + \sin\alpha \, \vec{e_2})}_{\widetilde{\vec{e_1}}}+
    \widetilde{y} \, \underbrace{(-\sin\alpha \, \vec{e_1} + \cos\alpha \, \vec{e_2})}_{\widetilde{\vec{e_{2}}}} \\
    & = & \underbrace{(\widetilde{x} \, \cos\alpha - \widetilde{y} \, \sin\alpha)}_{x} \, \vec{e_1}+
    \underbrace{(\widetilde{x} \, \sin\alpha + \widetilde{y} \, \cos\alpha)}_{y} \, \vec{e_2}
\end{array} \enn


\subsubsection{Polar coordinates}
Polar coordinates are often used if the problem under consideration has a certain symmetry. They are
represented by a vector $\vec{e_r}$ from the origin to a point in the plane and a vector $\vec{e_{\varphi}}$
from that point with the direction tangentially to the unit circle.
\bnn  \vec{S} = \left( \begin{array}{c} x \\  y \end{array} \right)
= x \, \vec{e_1} + y \, \vec{e_2}
=\left( \begin{array}{c} r \cos\varphi  \\  r \sin\varphi \end{array} \right)
=r \, \vec{e_r} + \varphi \, \vec{e_{\varphi}} 
\quad\rightarrow\quad\left( \begin{array}{c} r \\ \varphi \end{array} \right)_{pol}
\enn

\bnn \mbox{with}\quad r=\sqrt{x^{2}+y^{2}} \quad \varphi=\arctan \frac{y}{x} \enn \svs

\begin{figure}[!h]
    \centerline{\epsfxsize=10cm  \epsfbox{matlab/fig36.eps}}
    \caption{Polar coordinates}  \label{fig46}
\end{figure}

\vs {\bf Example:}
\bnn \vec{S}=\left( \begin{array}{c} 1 \\ 1 \end{array} \right)_{cart}= x \, \vec{e_1} + y \, \vec{e_2}  \qquad
\rightarrow \qquad r=\sqrt{1^{2}+1^{2}}=\sqrt{2} \qquad \varphi=\frac{\pi}{4}=45^0 \enn
\bnn 
\vec{S}=r \, \vec{e_r}+\frac{\pi}{4} \, \vec{e_{\varphi}} 
 \quad \rightarrow\quad \left(\begin{array}{c} \sqrt{2} \\ \frac{\pi}{4} \end{array} \right)_{pol}
\enn \svs

{\bf Note:} $\qquad \mbox{The quantity} \left(\begin{array}{c} \sqrt{2} \\ \frac{\pi}{4} \end{array} \right)_{pol}$
is {\em not} a vector!!!!

\subsubsection{Non-orthogonal Coordinate Systems}
Using a system of basis vectors that is orthogonal and normalized, i.e. 
$\vec{e_1} \cdot \vec{e_1}=\vec{e_2} \cdot \vec{e_2} =0$ and $\vec{e_1} \cdot \vec{e_2}=0$, or more general
$\vec{e_i} \cdot \vec{e_j}=\delta_{ij}$ is very convenient because it is straight forward to 
find a certain component of a vector simply by multiplying with the corresponding basis vector
\bnn
\vec{x} = x_1\,\vec{e_1} + x_2\,\vec{e_2} \; = \; \left\{ \begin{array}{c}     \vec{x}\cdot\vec{e_1}=x_1\,\overbrace{\vec{e_1}\cdot\vec{e_1}}^{=1}+\,x_2\,\overbrace{\vec{e_2}\cdot\vec{e_1}}^{=0}
=x_1 \svs \\        \vec{x}\cdot\vec{e_2}=x_1\,\underbrace{\vec{e_1}\cdot\vec{e_2}}_{=0}+\,x_2\,\underbrace{\vec{e_2}\cdot\vec{e_2}}_{=1}      
=x_2 \end{array} \right.
\enn

Sometimes, however it is necessary to represent vectors in a basis system $\vec{u},\,\vec{v}$ that is not orthogonal.
The easiest way to deal with this situation is to introduce a second set of basis vectors, the so-called {\em adjoint}
vectors or {\em dual basis} $\vec{u}^\dagger, \, \vec{v}^\dagger$ such the relations
\bnn
  \vec{u}^\dagger\cdot\vec{u}=\vec{v}^\dagger\cdot\vec{v} =1 \qquad\mbox{and}\qquad   
  \vec{u}^\dagger\cdot\vec{v}=\vec{v}^\dagger\cdot\vec{u} =0
\enn
are fulfilled. In components these equations read
\bnn
u_1^\dagger\,u_1+u_2^\dagger\,u_2=1 \qquad v_1^\dagger\,v_1+v_2^\dagger\,v_2=1 \qquad 
u_1^\dagger\,v_1+u_2^\dagger\,v_2=0 \qquad v_1^\dagger\,u_1+v_2^\dagger\,u_2=0
\enn
which are four equations for the four unknowns $u_i^\dagger,\,v_i^\dagger$ and allows us to determine the
adjoint vectors $\vec{u}^\dagger$ and $\vec{v}^\dagger$ if the original basis vectors $\vec{u}$ and $\vec{v}$
are linearly independent, i.e. not collinear.

Now we can express vectors in the basis $\vec{u}$ and $\vec{v}$, and determine their components by
multiplying with the corresponding vectors from the adjoint basis
\bnn
\vec{x} = a\,\vec{u} + b\,\vec{v} \; = \; \left\{ \begin{array}{c}     \vec{x}\cdot\vec{u}^\dagger=a\,\overbrace{\vec{u}\cdot\vec{u}^\dagger}^{=1}+\,b\,\overbrace{\vec{v}\cdot\vec{u}^\dagger}^{=0}
=a \svs \\        \vec{x}\cdot\vec{v}=a\,\underbrace{\vec{u}\cdot\vec{v}^\dagger}_{=0}+\,b\,\underbrace{\vec{v}\cdot\vec{v}^\dagger}_{=1}      
=b \end{array} \right.
\enn
 
%\begin{figure}[!h]
%    \centerline{\epsfxsize=7cm  \epsfbox{fig_dummy.eps}} \svs
%    \caption{Adjoint basis vectors}  \label{fig46}
%\end{figure}\vs

{\bf Note:} An ortho-normal basis system is simply the special case where the original and adjoint basis vectors are the same.


\subsection{Determinants}
The {\em determinant} is a is a descriptor of a matrix.
The determinant of a $2\times2$ matrix is given by
\bnn \det \mat{A}=\left| \begin{array}{cc} a_{11} & a_{12} \\ a_{21} & a_{22} \end{array} \right|
= a_{11} \, a_{22}-a_{21} \, a_{12} \enn

The determinant of a $3\times3$ matrix it is defined as
\bnn \det \mat{A}= \left| \begin{array}{ccc} a_{11} & a_{12} & a_{13} \\ a_{21} & a_{22} & a_{23} \\
  a_{31} & a_{32} & a_{33} \end{array} \right|
= a_{11} \, a_{22} \, a_{33} + a_{12} \, a_{23} \, a_{31} + a_{13} \, a_{21} \, a_{32}
- a_{13} \, a_{22} \, a_{31} - a_{11} \, a_{23} \, a_{32} -a_{33} \, a_{12} \, a_{21} \enn \vs

{\bf Hint:} The question arises, of course: "Who can remember something like this?" Well, it is actually
not that difficult using the following construction. First copy the left and the middle column to
the right. Then go through this scheme as indicated below: the left to right or southeast diagonals are
counted positive, the right to left or southwest diagonals are counted negative, resulting in the formula
for the determinant of a $3\times 3$ matrix (unfortunately such a procedure does not exist for higher
dimensional matrices and how to find their determinants is beyond the scope of this course).

\small
\bnn
\begin{array}{|ccc|cc}
a_{11} & a_{12} & a_{13} &  a_{11} & a_{12} \\
a_{21} & a_{22} & a_{23} &  a_{21} & a_{22} \\
a_{31} & a_{32} & a_{33} &  a_{31} & a_{32}  \end{array} \quad \Rightarrow \quad \left\{
\begin{array}{cccccccccl}
a_{11} & & a_{12} & & a_{13} & & a_{11} & & a_{12} & \\ & \searrow & & \searrow & & \searrow & & & & \\
a_{21} & & a_{22} & & a_{23} & & a_{21} & & a_{22} &  \qquad \mbox{\bf positive: +}
\\ & & & \searrow & & \searrow & & \searrow & & \\
a_{31} & & a_{32} & & a_{33} & & a_{31} & & a_{32} & \\ &&&&&&&&& \\ &&&&&&&&& \\ &&&&&&&&& \\
a_{11} & & a_{12} & & a_{13} & & a_{11} & & a_{12} \\ &  & & \swarrow & & \swarrow & & \swarrow & &\\
a_{21} & & a_{22} & & a_{23} & & a_{21} & & a_{22}  & \qquad \mbox{\bf negative: --}
\\ & \swarrow & & \swarrow & & \swarrow & & & & \\
a_{31} & & a_{32} & & a_{33} & & a_{31} & & a_{32} &
\end{array} \right.
\enn
\normalsize

{\bf Properties:}

If at least two of the column vectors are \emph{linearly dependent} the determinant $\det \mat{A}=0$. \\
\bnn \det (\mat{A} \, \mat{B})= \det \mat{A} \, \det \mat{B} \enn  \vs

{\bf Examples:}
\bnn \begin{array}{ccc} x-2y & = & \alpha_1 \\ 5 \, (x-2y) & = & \alpha_2 \end{array} \qquad
\mat{A} = \left( \begin{array}{cc} 1 & -2 \\ 5 & -10 \end{array} \right) \qquad
\vec{\alpha} = \left( \begin{array}{c} \alpha_1 \\ \alpha_2 \end{array} \right) \enn

\bnn \det \mat{A} =(-10) \cdot 1-(-2) \cdot 5=0 \enn \svs

\bnn \begin{array}{|rrr|rr}
3  & 1 & 2 &  \;3  & 1 \\
0 & -2 & 2 &  \;0 & -2 \\
1  & 3 & 0 &  \;1  & 3  
\end{array} = 3\cdot(-2)\cdot0+1\cdot2\cdot1+2\cdot0\cdot3-2\cdot(-2)\cdot1-3\cdot2\cdot3-1\cdot0\cdot0=-12
\enn \svs

{\bf Note:} The components of the vector product can be found from a determinant
\bnn
    \vec{a} \times \vec{b} = \left| \begin{array}{ccc}
    \vec{i} & \vec{j} & \vec{k} \\ a_1 & a_2 & a_3 \\ b_1 & b_2 & b_3 \end{array} \right|
    =\vec{i}\, (a_2\,b_3-a_3\,b_2) + \vec{j}\, (a_3\,b_1-a_1\,b_3) + \vec{k}\, (a_1\,b_2-a_2\,b_1)  
\enn

\subsection{The Inverse of a Matrix $\mat{A^{-1}}$}
The matrix $\mat{A}$ has an inverse with the property
$\mat{A} \, \mat{A^{-1}} = \mat{I} = \left( \begin{array}{cccc}
 1 & 0 & \hdots & 0 \\ 0 & 1 & \hdots & 0 \\ \vdots & \vdots & & \vdots \\ 0 & 0 & \hdots & 1
 \end{array} \right) \quad$ \mbox{if} $\det \mat{A} \neq 0$ \vs

{\bf Note:}
\bnn \vec{y}=\mat{A} \, \vec{x} \qquad \rightarrow \qquad \vec{x}=\mat{A^{-1}} \, \vec{y}  \enn
\bnn \det  (\mat{A} \, \mat{A^{-1}})= \det \mat{I}=1= \det \mat{A} \, \det \mat{A^{-1}} \qquad
\quad \rightarrow \quad \det \mat{A^{-1}}=\frac{1}{\det \mat{A}} \enn \svs

Inverse of a $2\times2 $ matrix:
\bnn \mat{A}= \left( \begin{array}{cc} a_{11} & a_{12} \\ a_{21} & a_{22} \end{array} \right)\qquad
\mat{A^{-1}}=\frac{1}{\det \: \mat{A}}\left( \begin{array}{rr} a_{22} & -a_{12} \\ -a_{21} & a_{11} \end{array} \right) \enn

\bnn
\Rightarrow \quad \mat{A} \, \mat{A^{-1}}  =\left( \begin{array}{cc} a_{11} & a_{12} \\ a_{21} & a_{22} \end{array} \right)
\frac{1}{\det \: \mat{A}}  \left( \begin{array}{rr} a_{22} & -a_{12} \\ -a_{21} & a_{11} \end{array} \right)
\qquad\qquad\qquad\qquad\qquad\qquad\qquad\qquad \enn

\bnn
 \qquad\qquad =\frac{1}{a_{11}\,a_{22}-a_{12}\,a_{21}}
    \left( \begin{array}{cc} a_{11}\,a_{22}-a_{12}\,a_{21} & 0 \\ 0 & a_{11}\,a_{22}-a_{12}\,a_{21} \end{array} \right)
= \mat{I} = \left( \begin{array}{cc} 1 & 0 \\ 0 & 1 \end{array} \right)
\enn \vs

\subsection{Linear Systems of Equations}
A system of the form
\bnn \begin{array}{ccc}
    y_1 & = & a_{11}\,x_1 + a_{12}\,x_2+ \, \hdots \, +a_{1n}\,x_n \\
    y_2 & = & a_{21}\,x_1 + a_{22}\,x_2+ \, \hdots \, +a_{2n}\,x_n \\
    \vdots & & \vdots \\
    y_n & = & a_{n1}\,x_1 + a_{n2}\,x_2+ \, \hdots \, +a_{nn}\,x_n \end{array}
    \qquad \mbox{or} \qquad
    \left( \begin{array}{c} y_1 \\ y_2 \\ \vdots \\ y_n \end{array} \right)
    = \left( \begin{array}{cccc} a_{11} & a_{12} & \hdots & a_{1n} \\ a_{21} & a_{22} & \hdots & a_{2n} \\
        \vdots & \vdots & \vdots & \vdots \\ a_{n1} & a_{n2} & \hdots &  a_{nn} \end{array} \right)
    \left( \begin{array}{c} x_1 \\ x_2 \\ \vdots \\ x_n \end{array} \right)
\enn

is called a {\em linear system of equations} and can be conveniently written in terms of 
vectors and matrices $\vec{y}=\mat{A}\,\vec{x}$. In most cases the coefficients $a_{ij}$ and
the values on the left hand side $y_i$ are known, and one is interested in finding a solution,
i.e. values for the $x_i$ such that all equations are fulfilled. What are the conditions that 
such a system has solutions and what are their properties?

We distinguish two cases:
\begin{enumerate}
\item $\vec{y} \neq \vec{0}$, i.e. at least one of the $y_i \neq 0$. In this case the system
       is called {\em inhomogeneous} and it has a unique solution if $\det \mat{A} \neq 0$. Then
       the matrix $\mat{A}$ has an inverse and the solution is given by $\vec{x}=\mat{A^{-1}}\, \vec{y}$.
       For $\det \mat{A} = 0$ the system has either no solution or infinitely many depending on $\vec{y}$;
\item $\vec{y} = \vec{0}$, i.e. all of the $y_i = 0$. In this case the system
       is called {\em homogeneous} and it has always the solution $\vec{x}=\vec{0}$, which is 
       called the {\em trivial} solution. Non-trivial solutions exist only if $\det \mat{A}=0$ and then
       there are infinitely many.
\end{enumerate}

{\bf Examples:}
\bnn
    \begin{array}{c} 3\,x_1 + x_2 = 6 \\ 3\,x_1 - x_2 =12 \end{array} 
    \quad \mbox{inhom.,} \quad 
    \det \mat{A} = \left| \begin{array}{cc} 3 & 1 \\ 3 & -1 \end{array} \right| = -6 \neq 0 \quad \rightarrow
    \quad \begin{array}{c} \mbox{unique} \\ \mbox{solution} \end{array} \quad \rightarrow
    \quad \begin{array}{c} x_1=3 \\ x_2= -3 \end{array}
\enn 
    
\bnn
    \begin{array}{c} 3\,x_1 + x_2 =6 \\  6\,x_1 + 2\,x_2 =10 \end{array} 
    \quad \mbox{inhom.,} \quad 
    \det \mat{A} = \left| \begin{array}{cc} 3 & 1 \\ 6 & 2 \end{array} \right| = 0  \quad \rightarrow
    \quad 12=10 \; \blitzd \quad \rightarrow \quad \begin{array}{c} \mbox{no} \\ \mbox{solution} \end{array}
\enn 
    
\bnn
    \begin{array}{c} 3\,x_1 + x_2 = 6 \\ 6\,x_1 +2\,x_2=12 \end{array} 
    \quad \mbox{inhom.,} \quad 
    \det \mat{A} = \left| \begin{array}{cc} 3 & 1 \\ 6 & 2 \end{array} \right| =  0 \quad \rightarrow
    \quad x_2=-3\,x_1+6 \quad \rightarrow \quad \begin{array}{c} \mbox{infinitely many} \\ \mbox{solutions} \end{array} 
\enn 
    
\bnn
    \begin{array}{c} 3\,x_1 + x_2 = 0 \\ 3\,x_1 - x_2 =0 \end{array} 
    \quad \mbox{hom.,} \quad 
    \det \mat{A} = \left| \begin{array}{cc} 3 & 1 \\ 3 & -1 \end{array} \right| = -6 \quad \rightarrow
    \quad \begin{array}{c} x_2=-3\,x_1 \\ x_2= 3\,x_1 \end{array} \quad \rightarrow
    \quad \begin{array}{c} \mbox{trivial} \\ \mbox{solution} \end{array} \quad \rightarrow \quad \vec{x}=\vec{0}
\enn 

\bnn
    \begin{array}{c} 6\,x_1 + 2\,x_2 = 0 \\ 3\,x_1 + x_2 =0 \end{array} 
    \quad \mbox{hom.,} \quad 
    \det \mat{A} = \left| \begin{array}{cc} 6 & 2 \\ 3 & 1 \end{array} \right| = 0 \quad \rightarrow
    \quad x_2=-3\,x_1 \quad \rightarrow
    \quad \begin{array}{c} \mbox{infinitely many} \\ \mbox{solution} \end{array} 
\enn 
     
\subsection{Eigenvalues and Eigenvectors}
A matrix performs a stretch, squeeze and/or rotation of a vector.The vector and matrix
elements depend on the choice of the coordinate system. Since this choice is arbitrary, the
question arises whether there is a special or canonical representation which is independent
of the coordinate system.

There are distinguished directions [eigendirections (eigen $\sim$ self)] along which a matrix operates.
Vectors pointing into these directions are only scaled but not rotated.

\bnn\label{eig} \mat{A} \vec{v}= \lambda \vec{v} \qquad \lambda \sim \mbox{eigenvalue}
\quad \vec{v} \sim \mbox{eigenvector} \enn
\bnn \mbox{or:} \quad  \underbrace{(\mat{A}-\lambda \, \mat{I})}_{\mat{B}} \, \vec{v}= 0 \qquad
\mbox{where} \quad \mat{I}=\left( \begin{array}{cc} 1 & 0 \\ 0 & 1 \end{array} \right)\mbox{ is the identity matrix.} \enn

The linear system of equations given by $\mat{b} \vec{v} =0$ is homogeneous and has nontrivial solutions
$\vec{v} \neq \vec{0}$ only if $\det \mat{b}=0$. The condition for non-vanishing eigenvectors is therefore 
given by
\bnn \det (\mat{A}-\lambda \, \mat{I})=0 \enn
from which the eigenvalues can be readily found. The eigenvector are then determined by solving
\bnn (\mat{A}-\lambda \, \mat{I}) \, \vec{v} = 0 \enn

\begin{figure}[!ht]
    \centerline{\epsfxsize=10cm  \epsfbox{matlab/fig37.eps}} \svs
    \caption{Eigenvalues and eigenvectors} \label{fig47}
\end{figure} \vs

{\bf Examples:} 
\bnn
\mat{A} = \left( \begin{array}{cc} 13 & 4 \\ 4 & 7 \end{array} \right) \qquad \rightarrow \quad \mbox{eigenvalues:}
\quad \det \, (\mat{A} -\lambda \, \mat{I}) = \left| \begin{array}{cc} 13-\lambda & 4 \\ 4 & 7-\lambda \end{array} \right|= 0
\enn
\bnn
\rightarrow \quad \lambda^2-20\lambda+75=0 \quad\mbox{(characteristic polynomial)} \qquad \rightarrow \quad
\lambda_1=15, \quad \lambda_2=5
\enn
\bnn
\lambda_1=15: \qquad \left. \begin{array}{ccccc} 13 \, v_1 & + & 4 \, v_2& = & 15 \, v_1 \\
4\, v_1 & + & 7\, v_2 & = & 15\,v_2 \end{array} \right. \qquad \rightarrow \quad v_1= 2 \, v_2
\enn
\bnn
\rightarrow \quad \mbox{choose:}\quad v_2=1 \quad \rightarrow \quad v_1=2 \quad \rightarrow \quad
\abs{\vec{v_1}}=\sqrt{5} \quad \Rightarrow \quad \vec{v_1}=\frac{1}{\sqrt{5}}
\left( \begin{array}{c} 2 \\ 1 \end{array} \right)
\enn
\bnn
\lambda_2=5: \;\;  \mat{A} \, \vec{v_2} = 5\, \vec{v_2} \;\;\rightarrow\;\; v_2=-2\, v_1 \;\;\rightarrow\;\;
\mbox{choose:} \;\; v_1=1 \;\;\rightarrow\;\;  v_2=-2 \;\;\Rightarrow\;\;
\vec{v_2}=\frac{1}{\sqrt{5}}\left( \begin{array}{c} 1 \\ -2 \end{array} \right)
\enn

\svs {\bf Note:} A matrix is called {\em symmetric} if $a_{ij}=a_{ji}$. Symmetric Matrices have real eigenvalues and
orthogonal eigenvectors $\vec{v_1} \cdot \vec{v_2} = 2\cdot 1 + 1 \cdot (-2) =0$.


\bnn
\mat{A} = \left( \begin{array}{cc} 1 & 0 \\ 2 & 2 \end{array} \right) \qquad \rightarrow \quad \mbox{eigenvalues:}
\quad \det \, (\mat{A} -\lambda \, \mat{I}) = \left| \begin{array}{cc} 1-\lambda & 0 \\ 2& 2-\lambda \end{array} \right|= 0
\enn
\bnn
\rightarrow \quad (1-\lambda)(2-\lambda)=0 \quad\mbox{(characteristic polynomial)} \qquad \rightarrow \quad
\lambda_1=1, \quad \lambda_2=2
\enn
\bnn
\rightarrow \quad \mbox{eigenvectors:} \quad \mat{A} \, \vec{v} = \lambda \, \vec{v} \qquad \rightarrow \quad
\left( \begin{array}{cc} 1 & 0 \\ 2 & 2 \end{array} \right) \left( \begin{array}{c} v_1 \\ v_2 \end{array} \right)
= \lambda \left( \begin{array}{c} v_1 \\ v_2 \end{array} \right)
\enn
\bnn
\lambda_1=1: \qquad \left. \begin{array}{ccc} v_1 & = & v_1 \\ 2\, v_1 + 2\, v_2 & = & v_2 \end{array} \right.
\qquad \rightarrow \quad v_1=- \frac{1}{2} v_2
\enn
\bnn
\rightarrow \quad \mbox{choose:}\quad v_2=2 \quad \rightarrow \quad v_1=-1 \quad \rightarrow \quad
\abs{\vec{v_1}}=\sqrt{5} \quad \Rightarrow \quad \vec{v_1}=\frac{1}{\sqrt{5}}
\left( \begin{array}{c} -1 \\ 2 \end{array} \right)
\enn
\bnn
\lambda_2=2: \;\;  \mat{A} \, \vec{v_2} = 2\, \vec{v_2} \;\;\rightarrow\;\; v_1=2\, v_1 \;\;\rightarrow\;\;
v_1=0 \;\;\rightarrow\;\; \mbox{choose:} \;\; v_2=1 \;\;\Rightarrow\;\;
\vec{v_2}=\left( \begin{array}{c} 0 \\ 1 \end{array} \right)
\enn \svs



\vs \begin{figure}[!h]
    \centerline{\epsfxsize=12cm  \epsfbox{matlab/fig38.eps}} \svs
    \caption{Determining eigenvalues and eigenvectors } \label{fig48}
\end{figure}

