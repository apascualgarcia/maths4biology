
% \setcounter{page}{12}

\section{Differential and Integral Calculus}\label{diff}
First derivatives of simple functions were studied by Galileo
Galilei (1564-1642) and Johannes Kepler (1571-1630). A systematic
theory of differential calculus was developed by Isaac Newton
(1643-1727) and Gottfried Wilhelm Leibniz (1646-1710).

\subsection{Difference Quotient}
The difference quotient becomes the differential in the limit
$h\rightarrow 0$ and describes the slope of a function $y=f(x)$
at a given point $x$.
\bnn y'(x)=\frac{dy}{dx}=\underset{h\rightarrow0}{\lim} \frac{y(x+h) - y(x)}{h} \enn

\begin{figure}[!h]
    \centerline{\epsfxsize=10cm \epsfysize=9cm \epsfbox{matlab/fig18.eps}} \svs
    \caption{The slope of a curve is found from its derivative.} \label{fig20}
\end{figure} \vs

\begin{figure}[!h]
    \centerline{\epsfxsize=10cm \epsfbox{matlab/fig19.eps}} \svs
    \caption{Slope as h$\rightarrow 0$.} \label{fig21}
\end{figure} \svs

{\bf Notation:} The limit value of the difference quotient is called the
derivative of a function $f(x)$. Derivatives are denoted by
\bnn y'(x)\;,\;\frac{dy}{dx}\;,\;\frac{df}{dx}\;,\;\frac{d}{dx}f(x) \qquad
\mbox{or sometimes in physics:} \;\; \dot{y}(t) \enn

{\bf Note:} Here we consider first-order derivatives only.

\vs
{\bf Example:}  $\qquad y = f(x) =x^2$
\bnn
y'\,=\,\frac{dy}{dx}=\underset{h\rightarrow 0}{\lim} \frac{(x+h)^2-x^2}{h}
\,=\,\underset{h\rightarrow 0}{\lim} \frac{x^2 +2hx +h^2 -x^2}{h}
\,=\,\underset{h\rightarrow 0}{\lim} \frac{2hx+h^2}{h} = 2x
\enn \vs

\subsection{Derivatives of Elementary Functions}

\subsubsection{Polynomials}
\vspace*{-2mm}\bnn y=x^2 \quad \rightarrow \quad \frac{dy}{dx}=2x
\qquad \qquad \qquad \mbox{more general:} \quad
y=x^n \quad \rightarrow \quad \frac{dy}{dx}=nx^{n-1} \enn

\subsubsection{Trigonometric functions}
\vspace*{-2mm}\bnn
y=\sin x \quad \rightarrow \quad \frac{dy}{dx}=\cos x \qquad \qquad \qquad
y=\cos x \quad \rightarrow \qquad \frac{dy}{dx}=-\sin x
\enn

\subsubsection{Exponential functions}
\vspace*{-2mm}\bnn y=e^x \quad \rightarrow \quad \frac{dy}{dx}=e^x \enn

\subsubsection{Hyperbolic functions}
\vspace*{-2mm}\bnn
y=\sinh x \quad \rightarrow \quad \frac{dy}{dx}=\cosh x \qquad \qquad \qquad
y=\cosh x \quad \rightarrow \quad \frac{dy}{dx}=\sinh x
\enn

\subsubsection{Logarithms}
\vspace*{-2mm}\bnn y=\ln x \quad \rightarrow \quad \frac{dy}{dx}=\frac{1}{x} \enn

\subsection{The Basic Rules for Calculating Derivatives}
If the derivatives of two functions $u(x)$ and $v(x)$ exist on an interval
$a<x<b$, then the derivatives of their combinations exist as well, i.e.
\bnn
u+v, \quad \alpha \, u \;\; \mbox{with} \;\; \alpha \in \mathbb{R}, \quad
u\,v, \quad \frac{u}{v} \;\; \mbox{if} \;\; v(x)\not=0 \;\; \mbox{for} \;\; a<x<b
\enn
{\bf Rules:}
\bnn \begin{array}{cc} \svs
\qquad (u + v)'=\frac{d}{dx}\{u+v\}=u' + v' & \qquad\qquad \mbox{\bf derivatives are additive} \qquad\qquad \\ \svs
\qquad (\alpha \, u)'=\frac{d}{dx}\{\alpha \, u\}=\alpha \, u' & 
\qquad\qquad \mbox{\bf multiplication with a scalar} \qquad\qquad \\ \svs
\qquad (u\,v)'=\frac{d}{dx}\{u\,v\}= u'\,v+u\,v'  & \qquad\qquad \mbox{\bf product rule} \qquad\qquad \\ \svs
\qquad (\frac{u}{v})'=\frac{d}{dx}\{\frac{u}{v}\}=\frac{u'\,v - u\,v'}{v^2} & \qquad\qquad \mbox{\bf quotient rule} \qquad\qquad
\end{array} \enn

{\bf Examples:}
\bnn \frac{d}{dx}\{x^{17}+\cos x\} = 17x^{16}-\sin x \enn
\bnn \frac{d}{dx}\{35 \cosh x\}=35 \frac{d}{dx}\,\cosh x=35 \sinh x \enn
\bnn \frac{d}{dx}\{\cos x e^x\} = - \sin x e^x +\cos x e^x = e^x (\cos x - \sin x) \enn
\bnn \frac{d}{dx}\{\frac{\cos x}{e^x}\} = \frac{-\sin x \, e^x - \cos x \, e^x}{e^{2x}}
    =\frac{-e^x \, (\sin x +\cos x)}{e^{2x}}= -\frac{\sin x + \cos x}{e^x} \enn \svs


\subsection{The Chain Rule}
If $u(x)$ and $v(x)$ have derivatives and the image of $v(x)$ is part of the source set of $u(x)$, 
then $u(v(x))$ has a derivative. 

To understand what this complicated sentence means, consider 
$\ln(\cos x)$. Here $u(x)=\ln x$ and $v(x)=\cos x$. The source set of $\cos x$ are all real numbers
$[-\infty, \infty]$, the image set of the cosine are the numbers in the interval [-1, 1], and the source 
set of the logarithm are all positive real numbers $]0, \infty]$. Therefore the image set of the cosine 
and the source set of the logarithm overlap in the interval $]0, 1]$. The source set of $\cos x$ that corresponds
to the image set $]0, 1]$ is given by all numbers where $\cos x$ is positive, i.e. $]-\frac{\pi}{2}, -\frac{\pi}{2}[$,
$]\frac{3\,\pi}{2}, -\frac{5\,\pi}{2}[$, etc., and the function $\ln(\cos x)$ exists and has a derivative for 
these values of $x$.
\bnn [u(v(x))]'=\frac{d}{dx}\{u(v(x))\}= \frac{d\,u(v)}{dv}\;\frac{d\,v(x)}{dx} \qquad\qquad \mbox{\bf chain rule} \enn

{\bf Examples:}
\bnn f(x)=\cos(\alpha x) \quad \rightarrow \quad u(v)=\cos v \quad \mbox{and} \quad v(x)=\alpha x \enn
\bnn \frac{d}{dx}\,\cos \alpha x=\frac{d\,\cos \alpha x}{d\,\alpha x}\;\frac{d\,\alpha x}{dx}
   =(-\sin \alpha x)\,\alpha = -\alpha \sin \alpha x \enn

\bnn f(x)=(2x+5)^3\quad \rightarrow \quad u(v)=v^3 \quad \mbox{and} \quad v(x)=2x+5 \enn
\bnn \frac{d}{dx}\,(2x+5)^3=\frac{d\,(2x+5)^3}{d\,(2x+5)}\,\frac{d\,(2x+5)}{dx}=3\,(2x+5)^2\;2=6\,(2x+5)^2 \enn

\subsection{Selected problems (the page from hell):}
{\bf Important note: Now we can take the derivative of ANY analytic function !!!}

\bnn f(x)=e^{\ln x} \quad \rightarrow \quad u(v)=e^v \quad \mbox{and} \quad v(x)=\ln x \enn
\bnn f'(x)=e^{\ln x}\,\frac{1}{x}=x\,\frac{1}{x}=1 \qquad \mbox{of course we started with}
  \quad f(x)=x \;\; \rightarrow \;\; f'(x)=1
\enn \svs

\bnn f(x)=\sqrt{\sin(3\,\alpha^2\,x^5)} = [\sin(3\,\alpha^2\,x^5)]^\frac{1}{2}=u(v(w(x))) \\
   \hspace*{3cm} \rightarrow \quad u(v)=v^\frac{1}{2} \quad v(w)=\sin(3\,\alpha^2\,x^5)
   \quad w(x)=3\,\alpha^2\,x^5 \enn
\bnn f'(x)=\frac{d\,u(v(w(x)))}{dv}\,\frac{d\,v(w(x))}{dw}\,\frac{d\,w(x)}{dx}
   =\frac{1}{2}\,[\sin(3\,\alpha^2\,x^5)]^{\frac{1}{2}-1} \; \cos(3\,\alpha^2\,x^5) \;
      3\,\alpha\,5\,x^{5-1} \\
   \hspace*{3cm} = \frac{15\,\alpha\,x^4\,\cos(3\alpha^2x^5)}{2\sqrt{\sin(3\alpha^2x^5)}}
      \qquad \mbox{who guessed this result ???}
\enn \svs

\bnn
 f(x)=\frac{3x^2+\cos kx}{\cosh x} \quad \rightarrow \quad
 f'(x)=\frac{(6x-k\sin kx)\cosh x + (3x^2+\cos kx)\sinh x}{\cosh^2x}
\enn

\hspace*{4cm}Also quite ugly, but technically correct !!! \vs

\bnn
f(x)=\cos^2 kx = \cos kx \, \cos kx \quad \rightarrow \quad f'(x)=2\,\cos kx (-\sin kx)\, k
   =-2k\,\cos kx \sin kx \\
   \hspace*{2cm} \mbox{or} \quad \rightarrow \quad (-\sin kx)\,k\,\cos kx+\cos kx (-\sin kx)\,k
       = -2k\,\cos kx \sin kx
\enn \svs

\bnn
f(x)=y=(x^5+e^{\cos kx})^{1/2} \quad \rightarrow \quad
  y'=\frac{1}{2}(x^5+e^{\cos kx})^{-1/2}\,(5\,x^4+e^{\cos kx}(-k\,\sin kx)) \\
    \hspace*{8cm} = \frac{5\,x^4-k\,\sin 2kx \, e^{\cos kx}}{2\,(x^5+e^{\cos kx})^{1/2}}
\enn \svs

\bnn
y=x^x=e^{x\ln x} \quad (\mbox{remember:} \; a^x=e^{x\ln a}) \quad \rightarrow \quad
y'=e^{x\ln x} \, (\ln x + 1) = x^x (\ln x + 1)
\enn \vs


\subsection{Integral Calculus: Definite Integrals}
How do you determine the area $A$ enclosed by a function $f(x)$
and the horizontal axis. It is simple if $f(x)=f_0=const$. \vs
\begin{figure}[!h]
    \centerline{\epsfxsize=12cm \epsfbox{matlab/fig20.eps}}
    \caption{Area $A$ enclosed by the horizontal axis and a horizontal line.} \label{fig22}
\end{figure} \vs

For the general case, divide the area $A$ into subareas
$A_{\nu}$ between $x_{\nu-1}$ and $x_{\nu}$. \vs
\begin{figure}[!h]
    \centerline{\epsfxsize=12cm  \epsfbox{matlab/fig21.eps}}
    \caption{Area enclosed by the function $f(x)$ and the horizontal axis.} \label{fig23}
\end{figure} \vs

Then the subarea $A_{\nu}$ may be approximated by
$A_{\nu}=f(\xi_{\nu}) ( x_{\nu}-x_{\nu-1} )$ for $x_{\nu-1}<\xi_{\nu}<x_{\nu}$.
There exists always a $\xi_{\nu}$ such that this is true.

Reconstruct the area $A$ as follows:
\bnn
A=\sum_{\nu=1}^N=A_{\nu}=\sum_{\nu=1}^N f(\xi_{\nu})\underbrace{(x_{\nu} - x_{\nu-1})}_{\Delta x}
\enn

This sum is called the {\em Riemann sum}. The limit of the Riemann sum is the area $A$ and defines the integral.
\bnn
A=\underset{N\rightarrow\infty}{\lim} \sum_{\nu=1}^N f(\xi_{\nu})\Delta x=\int_{x=a}^{x=b} \: f(x)\: dx
\enn

The area enclosed by $f(x)$ and the horizontal $x$-axis over an interval $x \in [a,b]$ is given by
definite integral
\bnn 
\int_a^b \: f(x)\, dx= F(x)\! \left.\frac{}{}\right|_{x=a}^{x=b} \,
= F(x)\! \left.\frac{}{}\right|_a^b \,= F(b)-F(a) 
\enn

where $F(x)$ is called the {\em anti-derivative} of $f(x)$ and
\bnn f(x) = \frac{dF(x)}{dx} = F'(x) \qquad\mbox{or, which is equivalent,}\qquad F(x)=\int \: f(x)\, dx+\mbox{const} \enn
Integration is to some extend the inverse operation of differentiation. \vs

{\bf Examples:}

\svs\begin{figure}[!h]
    \centering
    \subfigure[Definite integral of $y=x^2$]{\epsfxsize=7cm \epsfbox{matlab/fig22a.eps}}
    \hspace*{0.5cm}
    \subfigure[Definite integral of $y=x^3$]{\epsfxsize=7cm  \epsfbox{matlab/fig22b.eps}} \svs
    \caption{Definite integrals as areas under curves}  \label{fig24}
\end{figure} \svs

\begin{description}
\item[{\bf a)}]
The shaded area is given by
\bnn A=\int_{-1}^1 f(x)\,dx= F(1)-F(-1) \enn
We know $f(x)=x^2=F'(x)$. Now we guess $F(x)=1/3x^3+c$.

\bnn A= F(1)-F(-1)=\frac{1}{3} 1^3 + c - \{ \frac{1}{3} (-1)^3 + c \} = \frac{2}{3} \enn \vs

\item[{\bf b)}]
Again, the shaded area is given by
\bnn A=\int_{-1}^1 f(x)\,dx=F(1)-F(-1)\enn
We guess $F(x)=\frac{1}{4}x^4+c$ and find
\bnn A=\int_{-1}^1 f(x)\,dx=F(1)-F(-1) = \frac{1}{4}1^4 + c - \{ \frac{1}{4} (-1)^4 + c \} =0 \enn

Why does the area $A$ vanish? It actually consists of two areas, $A_1$ and $A_2$, which both have
the same size, but opposite sign $A_1=-A_2$.
\bnn A_1=F(0)-F(-1)=\frac{1}{4}0^4+c - \frac{1}{4}(-1)^4-c=-\frac{1}{4}=-A_2 \enn
\end{description}

{\bf Note:} In an integral the area {\em below} the x-axis is counted negative. In order to calculate the shaded area
we have to evaluate all pieces between intersections of the curve with the horizontal axis separately and add up their
magnitudes. Here: $A=\abs{A_1}+\abs{A_2}=\abs{-\frac{1}{4}}+\abs{\frac{1}{4}}=\frac{1}{2}$. \vs

\subsection{Methods of Integration}
{\bf Properties of integrals:}
\bnn \int_a^b f(x) \,dx = \int_a^c f(x) \:dx + \int_c^b f(x) \,dx \enn
\bnn \int_a^b f(x) \,dx = F(b)-F(a)=-(F(a)-F(b))=- \int_b^a f(x) \,dx \enn
\bnn \int_a^b (f_1(x)+f_2(x)) \:dx = \int_a^b \: f_1(x) \,dx + \int_a^b f(x)_2\,dx \enn
\bnn \int_a^b  c \, f(x) \,dx=c \,\int_a^b f(x) \,dx \qquad \mbox{where $c$ is a constant} \enn \vs

{\bf Approach (in order of preference):}
\begin{itemize}
\item {\bf Guess:} Find $F(x)$ such that $\frac{dF(x)}{dx} = f(x)$. For polynomials: $F(a\,x^n)=\frac{a}{n+1}\,x^{n+1}$.
\vspace*{2mm} \\ Note: Here $n$ can be negative or any rational number except -1. \\
\item {\bf Tables:} $F(x)$ can be looked up in mathematical tables of anti-derivatives and/or definite integrals
which can be found in e.g. Bronshteyn, Semendjajew or Gradsteyn.\\
\item {\bf Partial integration:} Corresponds to the product rule but only works for special cases.
\bnn \int f(x)\, g(x)dx = F(x)\, g(x)-\int \: F(x)\, g'(x)\,dx \enn
\bnn
\int_a^b x\, \cosh x \, dx = \underbrace{x}_{g}\underbrace{\sinh x}_{F} \, \arrowvert_a^b
   - \int_a^b\underbrace{\sinh x}_{F} \underbrace{1}_{g'}\, dx
   = b\sinh b - a \sinh a - (\cosh b - \cosh a)
\enn \svs
\item {\bf Substitution:} Corresponds to the chain rule but again only works for special cases.

\bnn
\int_{x=a}^{x=b} f(\phi(x)) \, \phi'(x)\, dx = \int_{u=\phi(a)}^{u=\phi(b)} f(u)\, du \qquad \mbox{where}\qquad u=\phi(x)
\enn \vs
\bnn \int_0^{\pi} \cos^2 x \; \sin x \, dx \quad \mbox{substitute:} \quad u=\cos x=\phi(x) \enn \vs
\bnn
  u'=\frac{du}{dx}=-\sin x = \phi'(x) \quad \rightarrow \quad
  du = -\sin x dx = \phi'(x) dx \quad \rightarrow\quad dx = -\frac{du}{\sin x}
\enn \vs

Substitute the integral:
\bnn
\int_{x=0}^{x=\pi} \cos^2 x \; \sin x \, \frac{-du}{\sin x}=-\int_{x=0}^{x=\pi} \cos^2 x \,du=-\int_{x=0}^{x=\pi} u^2 \,du
\enn \vs

Express the boundaries in terms of $u$:
\bnn x=0 \quad \rightarrow\quad u=\cos 0=1 \qquad\qquad\qquad\qquad x=\pi \quad \rightarrow\quad u=\cos \pi =-1 \enn \svs

Insert them and perform the integration:
\bnn
\int_0^{\pi} \cos^2 x \; \sin x \, dx
   =-\int_{u=1}^{u=-1} u^2 \, du = -\frac{1}{3}u^3 \! \left.\frac{}{}\right|_1^{-1} 
   = -\frac{1}{3}(-1)^3 + \frac{1}{3}1^3=\frac{2}{3}
\enn
\end{itemize} \vs

\subsection{Symmetries}
A function $f(x)$ is called an {\em even function} if $f(-x)=f(x)$; a function $g(x)$ is called an {\em odd function}
if $g(-x)=-g(x)$. The product of two even functions or the product of two odd functions is an even function; the product
of an odd and an even function is an odd function.

The integral over a symmetric interval around $x=0$ of an odd function vanishes.
 
\bnn 
\int_{-a}^{b=a} g(x) \, dx = \int_{-a}^{b=a} f(x) \, g(x) \: dx = 0 \qquad
\mbox{if} \;\; f(-x)=f(x) \quad \mbox{and} \quad g(-x)=-g(x) 
\enn

{\bf Example:}
\bnn \int_{-1}^1 \underbrace{x^2}_{f(x)} \underbrace{\sin 3x}_{g(x)}\: dx= 0 \enn

\begin{figure}[!h]
    \centerline{\epsfxsize=12cm \epsfysize=8cm  \epsfbox{matlab/fig23.eps}} \svs
    \caption{Due to symmetry the integral $\; \int_{-1}^1 x^2 \, \sin 3x\:dx\;$ vanishes} \label{fig26}
\end{figure} \vs \svs

\subsection{Orthogonality of trigonometric functions}
The cosine is an even function $\cos (-x)=\cos x$, and the sine is an odd function \mbox{$\sin(-x)=-\sin x$.}
Moreover, these trigonometric functions are $2\pi$-periodic, hence it is sufficient to consider integration
over windows of $2\pi$ only.
\bnn
\int_0^{2\pi}\underbrace{\sin x \, \cos x}_{\frac{1}{2}\sin 2x} \, dx
    = \frac{1}{2} \int_0^{2\pi}\sin 2x \, dx = -1/4\cos 2x \! \left.\frac{}{}\right|_0^{2\pi} = 0 \quad \mbox{or equivalent:} \;\;
     \int_{-\pi}^{\pi} \sin x \, \cos x\, dx = 0
\enn

\bnn
\int_{-\pi}^{\pi} \sin 2x \, \sin x \, dx = 2\int_{-\pi}^{\pi} \underbrace{\sin^2 x}_{u^2} \; \underbrace{\cos x \, dx}_{du}
    = 2 \int_{u=0}^{u=0} u^2 \: du = 0
\enn
Here we used the substitution $u=\sin x$ and $du=\cos x \, dx$ with the boundaries $x=\pi \rightarrow u=0$
and $x=-\pi \rightarrow u=0$. \svs

{\bf More general cases:}
\bnn \int_{-\pi}^{\pi} \cos mx \, \cos nx \, dx=\int_{-\pi}^{\pi} \sin mx \, \sin nx \, dx = \pi \, \delta_{mn} \enn
\bnn \int_{-\pi}^{\pi} \cos mx \sin nx \, dx = 0 \;\; \forall \;\; m,\,n \qquad\mbox{where "$\forall$" means "for all"} \enn

$\delta_{mn}$ is called the {\em Kronecker delta} which is defined as $\delta_{mn}=1$ if $m=n$ and $\delta_{mn}=0$ else.

\subsection{Integrals to Infinity}
If one or both boundaries of an integral are infinite this does not mean that the area under this curve
cannot be finite. A trivial example is given by the integral from $-\infty$ to $+\infty$ over an odd function.
This integral vanishes, as seen above, independent of the function as long as it is odd.

A nontrivial example is 
\bnn 
\int_1^{\infty} \frac{1}{x^2} \; dx = - \left. \frac{-1}{x} \right|_1^{\infty} 
    = -( \underbrace{\frac{1}{\infty}}_{=0} - \frac{1}{1}) = 1 
\enn
however
\bnn
\int_1^{\infty} \frac{1}{x} \; dx = \ln x \left. \frac{}{} \! \right|_1^{\infty} 
    = \ln \infty - \underbrace{\ln 1}_{=0} = \infty 
\enn \svs

In the same way even if a function has a singularity like $\frac{1}{\sqrt{x}}$ for $x \rightarrow 0$, 
the area can still be finite
\bnn
\int_0^2 x^{-\frac{1}{2}} \; dx = 2\, x^{\frac{1}{2}} \left. \frac{}{} \! \right|_0^2
    = 2\, \sqrt{x} \left. \frac{}{} \! \right|_0^2 
    = 2\,(\sqrt{2} - \sqrt{0}) = 2\,\sqrt{2} 
\enn
but again
\bnn
\int_0^2 x^{-1} \; dx = \int_0^2 \frac{1}{x} \; dx = \ln x \left. \frac{}{} \! \right|_0^2
    = \ln 2 - \underbrace{\ln 0}_{=-\infty} = \infty 
\enn \svs

And finally an exponential function
\bnn
\int_0^{\infty} e^{-x} \; dx = -e^{-x} \left. \frac{}{} \! \right|_0^{\infty}
    = -(\underbrace{e^{-\infty}}_{=0}-\underbrace{e^0}_{=1}) = 1 
\enn

\begin{figure}[!h]
    \centering
    \subfigure[$\int_1^{\infty} x^{-2} \, dx = 1$]{\epsfxsize=4.5cm \epsfbox{matlab/fig23_1a.eps}}
    \hspace*{0.5cm}
    \subfigure[$\int_0^2 x^{-\frac{1}{2}} \, dx = 2\,\sqrt{2}$]{\epsfxsize=4.5cm \epsfbox{matlab/fig23_1b.eps}} 
    \hspace*{0.5cm}
    \subfigure[$\int_0^{\infty} e^{-x} \, dx = 1$]{\epsfxsize=4.5cm \epsfbox{matlab/fig23_1c.eps}} \\ \vs
    \hspace*{0.5cm}
    \subfigure[$\int_1^{\infty} x^{-1} \, dx = \infty$]{\epsfxsize=4.5cm \epsfbox{matlab/fig23_1d.eps}}
    \hspace*{0.5cm}
    \subfigure[$\int_0^2 x^{-1} \, dx = \infty$]{\epsfxsize=4.5cm \epsfbox{matlab/fig23_1e.eps}} \svs
    \caption{Definite integrals that involve infinities}  \label{fig24}
\end{figure} \vs

\subsection{Functions with no Antiderivative}
As we have seen, it quite straightforward to calculate the derivatives of quite complicated
"monsters" of functions. On the other hand, it is much more difficult to find antiderivatives.
To make things worse there are certain functions with very important applications for which
an antiderivative does not exist, i.e. it cannot be expressed in terms of elementary functions.

One of these simple functions which do not have an antiderivative is $f(x)=e^{-x^2}$. This 
is very inconvenient because this function is the famous bell-shaped Gaussian which rules the
entire field of statistics, because the probability that an event occurs within a certain interval
of a parameter is given by area under this curve. This area unfortunately cannot be calculated
using a pocket calculator that has only elementary functions. The definite integral can be 
found numerically or looked up in tables, and it also has a name: the "error function" erf$(x)$
\bnn  
\mbox{erf}(x)=\frac{2}{\sqrt{\pi}} \, \int_0^x e^{-u^2} \, du  \qquad\qquad \mbox{note:}\qquad
\int_{-\infty}^{\infty} e^{-x^2}\,dx = \sqrt{\pi}
\enn  \svs

A second example of such a function with no antiderivative is the so-called "integral sine" Si$(x)$
\bnn 
\mbox{Si}(x)=\int_0^x \frac{\sin u}{u} \, du  \qquad\qquad \mbox{note:}\qquad
\int_{-\infty}^{\infty} \frac{\sin x}{x} \, dx = \pi
\enn  
\begin{figure}[t]
    \centerline{
    \subfigure[$\int_{-\infty}^{\infty} e^{-x^2}\,dx=\sqrt{\pi}$]{\epsfxsize=7cm \epsfbox{matlab/fig23_2a.eps}}
    \hspace*{0.5cm}
    \subfigure[$\int_{-\infty}^{\infty}{\frac{\sin x}{x}}\,dx=\pi$]{\epsfxsize=7cm \epsfbox{matlab/fig23_2b.eps}}} \svs 
    \caption{Definite integrals over functions with no antiderivatives}  \label{fig24}
\end{figure}







  