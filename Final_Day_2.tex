
% \setcounter{page}{12}

\section{Limits and Derivatives}\label{diff}
First derivatives of simple functions were studied by Galileo
Galilei (1564-1642) and Johannes Kepler (1571-1630). A systematic
theory of differential calculus was developed by Isaac Newton
(1643-1727) and Gottfried Wilhelm Leibniz (1646-1710).

\subsection{Difference Quotient}
The difference quotient becomes the differential in the limit
$h\rightarrow 0$ and describes the slope of a function $y=f(x)$
at a given point $x$.
\bnn y'(x)=\frac{dy}{dx}=\underset{h\rightarrow0}{\lim} \frac{y(x+h) - y(x)}{h} \enn

\begin{figure}[!h]
    \centerline{\epsfxsize=10cm \epsfysize=9cm \epsfbox{matlab/fig18.eps}} \svs
    \caption{The slope of a curve is found from its derivative.} \label{fig20}
\end{figure} \vs

\begin{figure}[!h]
    \centerline{\epsfxsize=10cm \epsfbox{matlab/fig19.eps}} \svs
    \caption{Slope as h$\rightarrow 0$.} \label{fig21}
\end{figure} \svs

{\bf Notation:} The limit value of the difference quotient is called the
derivative of a function $f(x)$. Derivatives are denoted by
\bnn y'(x)\;,\;\frac{dy}{dx}\;,\;\frac{df}{dx}\;,\;\frac{d}{dx}f(x) \qquad
\mbox{or sometimes in physics:} \;\; \dot{y}(t) \enn

{\bf Note:} Here we consider first-order derivatives only.

\vs
{\bf Example:}  $\qquad y = f(x) =x^2$
\bnn
y'\,=\,\frac{dy}{dx}=\underset{h\rightarrow 0}{\lim} \frac{(x+h)^2-x^2}{h}
\,=\,\underset{h\rightarrow 0}{\lim} \frac{x^2 +2hx +h^2 -x^2}{h}
\,=\,\underset{h\rightarrow 0}{\lim} \frac{2hx+h^2}{h} = 2x
\enn \vs

\subsection{Derivatives of Elementary Functions}

\subsubsection{Polynomials}
\vspace*{-2mm}\bnn y=x^2 \quad \rightarrow \quad \frac{dy}{dx}=2x
\qquad \qquad \qquad \mbox{more general:} \quad
y=x^n \quad \rightarrow \quad \frac{dy}{dx}=nx^{n-1} \enn

\subsubsection{Trigonometric functions}
\vspace*{-2mm}\bnn
y=\sin x \quad \rightarrow \quad \frac{dy}{dx}=\cos x \qquad \qquad \qquad
y=\cos x \quad \rightarrow \qquad \frac{dy}{dx}=-\sin x
\enn

\subsubsection{Exponential functions}
\vspace*{-2mm}\bnn y=e^x \quad \rightarrow \quad \frac{dy}{dx}=e^x \enn

\subsubsection{Hyperbolic functions}
\vspace*{-2mm}\bnn
y=\sinh x \quad \rightarrow \quad \frac{dy}{dx}=\cosh x \qquad \qquad \qquad
y=\cosh x \quad \rightarrow \quad \frac{dy}{dx}=\sinh x
\enn

\subsubsection{Logarithms}
\vspace*{-2mm}\bnn y=\ln x \quad \rightarrow \quad \frac{dy}{dx}=\frac{1}{x} \enn

\subsection{The Basic Rules for Calculating Derivatives}
If the derivatives of two functions $u(x)$ and $v(x)$ exist on an interval
$a<x<b$, then the derivatives of their combinations exist as well, i.e.
\bnn
u+v, \quad \alpha \, u \;\; \mbox{with} \;\; \alpha \in \mathbb{R}, \quad
u\,v, \quad \frac{u}{v} \;\; \mbox{if} \;\; v(x)\not=0 \;\; \mbox{for} \;\; a<x<b
\enn
{\bf Rules:}
\bnn \begin{array}{cc} \svs
\qquad (u + v)'=\frac{d}{dx}\{u+v\}=u' + v' & \qquad\qquad \mbox{\bf derivatives are additive} \qquad\qquad \\ \svs
\qquad (\alpha \, u)'=\frac{d}{dx}\{\alpha \, u\}=\alpha \, u' & 
\qquad\qquad \mbox{\bf multiplication with a scalar} \qquad\qquad \\ \svs
\qquad (u\,v)'=\frac{d}{dx}\{u\,v\}= u'\,v+u\,v'  & \qquad\qquad \mbox{\bf product rule} \qquad\qquad \\ \svs
\qquad (\frac{u}{v})'=\frac{d}{dx}\{\frac{u}{v}\}=\frac{u'\,v - u\,v'}{v^2} & \qquad\qquad \mbox{\bf quotient rule} \qquad\qquad
\end{array} \enn

{\bf Examples:}
\bnn \frac{d}{dx}\{x^{17}+\cos x\} = 17x^{16}-\sin x \enn
\bnn \frac{d}{dx}\{35 \cosh x\}=35 \frac{d}{dx}\,\cosh x=35 \sinh x \enn
\bnn \frac{d}{dx}\{\cos x e^x\} = - \sin x e^x +\cos x e^x = e^x (\cos x - \sin x) \enn
\bnn \frac{d}{dx}\{\frac{\cos x}{e^x}\} = \frac{-\sin x \, e^x - \cos x \, e^x}{e^{2x}}
    =\frac{-e^x \, (\sin x +\cos x)}{e^{2x}}= -\frac{\sin x + \cos x}{e^x} \enn \svs


\subsection{The Chain Rule}
If $u(x)$ and $v(x)$ have derivatives and the image of $v(x)$ is part of the source set of $u(x)$, 
then $u(v(x))$ has a derivative. 

To understand what this complicated sentence means, consider 
$\ln(\cos x)$. Here $u(x)=\ln x$ and $v(x)=\cos x$. The source set of $\cos x$ are all real numbers
$[-\infty, \infty]$, the image set of the cosine are the numbers in the interval [-1, 1], and the source 
set of the logarithm are all positive real numbers $]0, \infty]$. Therefore the image set of the cosine 
and the source set of the logarithm overlap in the interval $]0, 1]$. The source set of $\cos x$ that corresponds
to the image set $]0, 1]$ is given by all numbers where $\cos x$ is positive, i.e. $]-\frac{\pi}{2}, -\frac{\pi}{2}[$,
$]\frac{3\,\pi}{2}, -\frac{5\,\pi}{2}[$, etc., and the function $\ln(\cos x)$ exists and has a derivative for 
these values of $x$.
\bnn [u(v(x))]'=\frac{d}{dx}\{u(v(x))\}= \frac{d\,u(v)}{dv}\;\frac{d\,v(x)}{dx} \qquad\qquad \mbox{\bf chain rule} \enn

{\bf Examples:}
\bnn f(x)=\cos(\alpha x) \quad \rightarrow \quad u(v)=\cos v \quad \mbox{and} \quad v(x)=\alpha x \enn
\bnn \frac{d}{dx}\,\cos \alpha x=\frac{d\,\cos \alpha x}{d\,\alpha x}\;\frac{d\,\alpha x}{dx}
   =(-\sin \alpha x)\,\alpha = -\alpha \sin \alpha x \enn

\bnn f(x)=(2x+5)^3\quad \rightarrow \quad u(v)=v^3 \quad \mbox{and} \quad v(x)=2x+5 \enn
\bnn \frac{d}{dx}\,(2x+5)^3=\frac{d\,(2x+5)^3}{d\,(2x+5)}\,\frac{d\,(2x+5)}{dx}=3\,(2x+5)^2\;2=6\,(2x+5)^2 \enn

\subsection{Selected problems (the page from hell):}
{\bf Important note: Now we can take the derivative of ANY analytic function !!!}

\bnn f(x)=e^{\ln x} \quad \rightarrow \quad u(v)=e^v \quad \mbox{and} \quad v(x)=\ln x \enn
\bnn f'(x)=e^{\ln x}\,\frac{1}{x}=x\,\frac{1}{x}=1 \qquad \mbox{of course we started with}
  \quad f(x)=x \;\; \rightarrow \;\; f'(x)=1
\enn \svs

\bnn f(x)=\sqrt{\sin(3\,\alpha^2\,x^5)} = [\sin(3\,\alpha^2\,x^5)]^\frac{1}{2}=u(v(w(x))) \\
   \hspace*{3cm} \rightarrow \quad u(v)=v^\frac{1}{2} \quad v(w)=\sin(3\,\alpha^2\,x^5)
   \quad w(x)=3\,\alpha^2\,x^5 \enn
\bnn f'(x)=\frac{d\,u(v(w(x)))}{dv}\,\frac{d\,v(w(x))}{dw}\,\frac{d\,w(x)}{dx}
   =\frac{1}{2}\,[\sin(3\,\alpha^2\,x^5)]^{\frac{1}{2}-1} \; \cos(3\,\alpha^2\,x^5) \;
      3\,\alpha\,5\,x^{5-1} \\
   \hspace*{3cm} = \frac{15\,\alpha\,x^4\,\cos(3\alpha^2x^5)}{2\sqrt{\sin(3\alpha^2x^5)}}
      \qquad \mbox{who guessed this result ???}
\enn \svs

\bnn
 f(x)=\frac{3x^2+\cos kx}{\cosh x} \quad \rightarrow \quad
 f'(x)=\frac{(6x-k\sin kx)\cosh x + (3x^2+\cos kx)\sinh x}{\cosh^2x}
\enn

\hspace*{4cm}Also quite ugly, but technically correct !!! \vs

\bnn
f(x)=\cos^2 kx = \cos kx \, \cos kx \quad \rightarrow \quad f'(x)=2\,\cos kx (-\sin kx)\, k
   =-2k\,\cos kx \sin kx \\
   \hspace*{2cm} \mbox{or} \quad \rightarrow \quad (-\sin kx)\,k\,\cos kx+\cos kx (-\sin kx)\,k
       = -2k\,\cos kx \sin kx
\enn \svs

\bnn
f(x)=y=(x^5+e^{\cos kx})^{1/2} \quad \rightarrow \quad
  y'=\frac{1}{2}(x^5+e^{\cos kx})^{-1/2}\,(5\,x^4+e^{\cos kx}(-k\,\sin kx)) \\
    \hspace*{8cm} = \frac{5\,x^4-k\,\sin 2kx \, e^{\cos kx}}{2\,(x^5+e^{\cos kx})^{1/2}}
\enn \svs

\bnn
y=x^x=e^{x\ln x} \quad (\mbox{remember:} \; a^x=e^{x\ln a}) \quad \rightarrow \quad
y'=e^{x\ln x} \, (\ln x + 1) = x^x (\ln x + 1)
\enn \vs




  