\documentclass[11pt]{article}
\usepackage{makeidx}
\usepackage{latexsym}
\usepackage{color}
\usepackage{wrapfig}
\usepackage{here}
\usepackage{graphicx}
\usepackage[centertags]{amsmath}
\usepackage{amsfonts}
\usepackage{amssymb}
\usepackage{amsthm}
\usepackage{subfigure}
\usepackage{here}

\addtolength{\topmargin}{-2.5cm} \addtolength{\textheight}{4.5cm}
\addtolength{\oddsidemargin}{-2cm} \addtolength{\textwidth}{3.8cm}
\addtolength{\evensidemargin}{-2cm}

\begin{document}
\renewcommand\floatpagefraction{.9}
\renewcommand\topfraction{.9}
\renewcommand\bottomfraction{.9}
\renewcommand\textfraction{.1}
\setcounter{totalnumber}{50} \setcounter{topnumber}{50}
\setcounter{bottomnumber}{50} \floatsep 20 pt \intextsep 5 pt
\abovecaptionskip 0 pt
\belowcaptionskip 0 pt

\newcommand{\beq}{\begin{equation}}
\newcommand{\eeq}{\end{equation}}
\newcommand{\ba}{\begin{array}}
\newcommand{\ea}{\end{array}}
\newcommand{\bea}{\begin{eqnarray}}
\newcommand{\eea}{\end{eqnarray}}
\newcommand{\bes}{\begin{eqnarray*}}
\newcommand{\ees}{\end{eqnarray*}}
\newcommand{\vs}{\vspace*{0.5cm}}
\newcommand{\svs}{\vspace*{0.25cm}}

\newcommand{\abs}[1]{\mid \! #1\! \mid}
\renewcommand{\b}[1]{\mbox{{\bf #1}}}
\renewcommand{\vec}[1]{\protect\overrightarrow{ #1}}
\newcommand{\mat}[1]{\mbox{\boldmath $#1$}}

\long\def\bform#1\eform{\begin{equation}
   \begin{minipage}{\eqbreite}\vskip .1cm
   \let\\ = \thcr
   \halign{$\displaystyle{##}$ \hfil
  && $\displaystyle{##}$\hfil\cr#1\cr}
   \vskip .3cm \end{minipage}
\end{equation}}
\def\thcr{\cr\noalign{\vskip.3cm}}
\newlength{\eqbreite}\setlength{\eqbreite}{\textwidth}
\addtolength{\eqbreite}{-1.5cm}

\long\def\bnn#1\enn{$$\begin{minipage}{\eqbreite}
\vskip .1cm \let\\ = \thcr \halign{$\displaystyle{##}$
\hfil && $\dispaystyle{##}$\hfil \cr#1 \cr}
\vskip.3cm \end{minipage}$$}

\parindent 0 in
\parskip .37 cm

\input epsf 

\pagestyle{empty}

\LARGE
\begin{center}
\underline{\textbf{ Complex Systems Boot Camp }}
\end{center}

\begin{flushleft}
\large \verb"ISC6930"\hspace{5.1cm} \textsc{Section}\#1  \hspace{4.cm} \textsc{Date:} \ldots\ldots. \\
\vspace*{.3cm} \large \textsc{Name:}
\ldots\ldots\ldots\ldots\ldots\ldots\hspace{1.85cm} SS\#:
\ldots\ldots\ldots\ldots\ldots\ldots\hspace{2cm}\textsc{Points:} \ldots\ldots\\
\line(1,0){455}
\end{flushleft}

\normalsize
\textbf{1)} Sketch the polynomial functions:
\bnn\begin{array}{lll}
\mbox{a)}\quad p(x) = x^2  \qquad  & \mbox{b)} \quad  p(x) =( x - 2)^2  \qquad   & \mbox{c)} \quad   p(x) = (x + 4)^2 \vs \\ 
\mbox{d)}\quad p(x)=x^2-9  \qquad  & \mbox{e)} \quad  p(x)=\frac{1}{2}\,x^{2}  \qquad & \mbox{f)}\quad p(x) =3\,(x-2)^2 \vs \\
\mbox{g)}\quad p(x) = 3\,(x - 2)^2 - 5  \qquad & \mbox{h)}\quad  p(x) = -2\,x^2  \qquad & \mbox{i)}\quad  p(x) = -x^2 + 5 \svs\\
\mbox{j)}\quad p(x) = x^3+x^2-6\,x \quad & \mbox{k)}\quad  p(x)=(x + 4)^3  \qquad   & \mbox{l)}\quad   p(x)=x-x^3
\end{array} \enn

\textbf{2)} Which of the graphs in the problem (1) have the same
exact shape even though they lie in \\
\hspace*{4mm} different locations of the coordinate plane.

\textbf{3)} Without graphing determine whether the polynomial
$-2\,x^{10} + \pi \, x^{7} - e \, x^2 + 6$ opens up or down.

\textbf{4)} Without graphing determine whether the polynomial
$-2\,x^7 - x^2 + 6$ rises to left or right.

\textbf{5)} What are the roots  of the polynomial: 
$\pi \, (x - e)^5 \, (x + 3)^2\,(4x - 11)\,x$

\textbf{6)} Sketch the trigonometric functions:
\bnn \begin{array}{lll}
\mbox{a)}\quad f(x)=-\sin x \qquad & \mbox{b)}\quad f(x)=2\,\sin x \qquad & \mbox{c)}\qquad f(x)=\frac{1}{2}\,\sin x \svs \\
\mbox{d)}\quad f(x)=\sin x\; (x-\frac{\pi}{4})\qquad & \mbox{e)}\quad f(x)=\sin2x \qquad  & \mbox{f)}\quad f(x)=\sin x\;(2\,x + \frac{\pi}{2}) \svs \\
\mbox{g)} \quad f(x)=3+\sin x  \quad                &                                   & 
\end{array}\enn

\textbf{7)} Find the amplitude, period, frequency and phase of
$ f(x) = \sqrt{26} \, \sin(\pi \, x + \frac{\pi^2}{3})$.

\textbf{8)} Describe how the graph of
 $f(x) = 6 \, \cos(\frac{1}{3}\,x + \frac{\pi}{9}) - 7$ can be obtained geometrically from the \\
 \hspace*{4mm} graph of the function $f(x) = \cos x$.

\textbf{9)} The graph of $f(x) = \pi \, \cos 7x$ is the same as the
graph of: $\; \mbox{a)}\;\; \pi\, \sin(-7x)  \quad  \mbox{b)} \;\; \pi \, \cos (-7x)$

\textbf{10)} Sketch\hspace*{1cm}$ f(x) = x + \sin x$

\textbf{11)} Sketch\hspace*{1cm}$ f(x) = e^{2\,x}$.

\textbf{12)} Sketch\hspace*{1cm}$ f(x) = 3 - e^x$.

\textbf{13)} Sketch\hspace*{1cm}$ f(x) = \ln(x + 2)$.

\textbf{14)} Describe how the graph of
$f(x) = 2 \, \sinh(x - 3)$ can be obtained geometrically from the graph \\
\hspace*{7mm} of the function $f(x) = \sinh x$.

\textbf{15) }Calculate $\quad \mbox{a)} \quad \ln \, e^{\pi} \qquad
\mbox{b)}\quad \ln 1$.

\end{document}
