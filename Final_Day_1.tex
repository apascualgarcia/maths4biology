%\setcounter{page}{1}

\section{Function Theory}
\subsection{Foundations}

{\bf Definition:} Consider two sets of elements $A$ and $B$. A {\em function} $f$ is a rule which unambiguously
assigns $y \in B$ to each $x \in A$. We call the set $A$ the {\em domain} of the function and the set $B$ its
{\em codomain}.

{\bf Note:} "$x \in A$" means that "$x$ is an element of $A$".\vs

\begin{figure}[!h]
    \centerline{\epsfxsize=10cm  \epsfbox{matlab/fig1.eps}} \vs
    \caption{Illustration of a function} \label{fig1}
\end{figure} \vs

{\bf Notation:}
\bnn \begin{array}{ccc}
    f: & A \rightarrow B & \\ \svs
    f: & A \rightarrow B & \;\; x\in A, \; y\in B \\ \svs
    & y = f(x) & \;\; x\in A, \; y\in B \\ \svs
    & y = y(x) & \;\; x\in A, \; y\in B
\end{array} \enn \vs

{\bf Examples:}

\vs \begin{figure}[!h]
   \centerline{\epsfxsize=10cm \epsfbox{matlab/fig2.eps}} \vs
   \caption{Assigning an element $\{21\} \in B$ to $\{1\} \in A$} \label{fig2}
\end{figure} \vs

\newpage

Most commonly, functions are defined by equations: \qquad a) $ y = f(x) = 2x + 1$, \qquad b) $y=f(x)=x^2$  \svs

{Graphical representation:} \svs
\begin{figure}[!h]
    \centering
    \subfigure[Function $y = 2x + 1$]{\epsfxsize=7cm \epsfbox{matlab/fig3a.eps}}
    \hspace*{0.5cm}
    \subfigure[Function $y = x^2$]{\epsfxsize=7cm \epsfbox{matlab/fig3b.eps}} \svs
    \caption{A linear and a quadratic function} \label{fig3}
\end{figure} \vs

{\bf Types:} Some important types of functions are:
\begin{enumerate}
\item Injective functions: Those functions that preserve distinctiveness, i.e. they never map distinct elements of its domain to the same elements of its codomain.
\item Surjective functions: Every element $y$ of the codomain has a correspondent element $x$ of the domain that is mapping it.
\item Bijective functions: Those functions that are both injective and surjective.
\end{enumerate}

\subsection{Domain, parity and periodicity of functions}

{\bf Domain of a function} is the set of input values in which the function can be defined. Therefore, given
a map between two sets, to define a function we should determine a valid domain identifying possible {\em pathological} input values.

{\bf Example:} 
The map $f(x) = 1/x$ has a pathological value at $x=0$ meaning that the map diverges to infinity. In order
for this map to be a function, we should {\em restrict the domain} and define the function as:

\bnn 
  f(x)=\begin{cases}
    1/x, & \text{if $x\neq0$}.\\
    0, & \text{if $x=0$}.
  \end{cases}
\enn

{\bf Parity of a function.} Let $f(x)$ be a real-valued function, then:
\begin{enumerate}
 \item $f$ is {\em even} if the following equation holds for all $x$ in the domain of $f$:
 		\bnn f(x) = f(-x) \quad \rightarrow \quad f(x)-f(-x)=0 \enn
 \item $f$ is {\em odd} if
 	    \bnn -f(x) = f(-x) \quad  \rightarrow \quad f(x)+f(-x)=0.\enn
\end{enumerate}

{\bf Note:} These properties will be very useful to guess some properties of derivatives and integrals of functions.

{\bf Periodicity of a function.} We say that a function $f(x)$ is periodic if for all $x$ in the domain of $f$ there is a value $\lambda \in \mathbb{R}$ such that for a every $n \in \mathbb{N}$ it holds that $f(x) = f(x+n\lambda)$, where $\lambda$ is the {\em period} of the function.

\subsection{Inverse Functions}

$f^{-1}$ denotes the inverse of the function $f$. \vs

\begin{figure}[!h]
   \centerline{\epsfxsize=11cm  \epsfbox{matlab/fig4.eps}} \svs
    \caption{Inverse function} \label{fig4}
\end{figure} \vs

{\bf Notation:}
\bnn f^{-1}: \; B \rightarrow A \enn \svs
\bnn x = f^{-1}\,(y) \qquad \mbox{where} \qquad y = f(x) \enn \svs

Graphically the inverse can be constructed as the mirror image of the function at the
first bisector. This method always works, but caution is asked for, because the inverse 
may not be unique and require more detailed discussion.

{\bf Example:}
\bnn y = f(x) = 2x + 1  \qquad  x = f^{-1}\,(y) = \frac{1}{2} \, (y-1) \enn

{\bf Note:} There is not always an inverse function!  \vs

\begin{figure}[!h]
   \centerline{\epsfxsize=11cm \epsfbox{matlab/fig5.eps}} \svs
    \caption{The inverse $f^{-1}$ is not unique, thus not a function.} \label{fig5}
\end{figure} \vs


{\bf Example:}
\bnn y = f(x) = x^2 \enn
\bnn x = \sqrt{y} \qquad \mbox{or} \qquad x = -\sqrt{y} \enn  \vs

\begin{figure}[!h]
    \centering
    \subfigure[$\; f(x)=2\,x+1 \; \rightarrow \; f^{-1}(x)=\frac{1}{2}(x-1)$]{\epsfxsize=7cm \epsfysize=7cm
        \epsfbox{matlab/fig6a.eps}}
    \hspace*{0.5cm}
    \subfigure[$\; f(x)=x^2 \; \rightarrow \; f^{-1}(x)=\pm\sqrt{x}$]{\epsfxsize=7cm \epsfysize=7cm
         \epsfbox{matlab/fig6b.eps}} \svs
    \caption{Graphical construction of the inverse function} \label{fig6}
\end{figure}

\newpage
\subsection{Implicit Functions}
A function is not given explicitly as in $y = f(x)$, but
implicitly by $F(x,y) = 0$. Note that an implicit equation may not necessarily fulfill
the conditions of a function when written explicitly, unless some additional conditions
are imposed.

{\bf Example:} Unit circle: $\quad F(x,y)=x^2+y^2-1=0$. In the plot, we observe that
each $x$ point corresponds to two $y$ points and hence it is not a function. \svs

\begin{figure}[!h]
    \centerline{\epsfxsize=7cm \epsfysize=7cm \epsfbox{matlab/fig7.eps}} \svs
    \caption{Unit circle} \label{fig7}
\end{figure} \svs

The implicit representation of the unit circle needs additional conditions to become
unique and thus a function: a local neighborhood has to be defined, e.g. $y=\sqrt{1-x^2}$
for \mbox{$x\in (-1;1)$},  \mbox{$y >0$} and $y=-\sqrt{1-x^2}$ for $x\in (-1;1), \, y<0$.

{\bf Note:} The implicit representation is particularly important for algebraic equations
of the form:

\bnn
a_n(x)y^n + a_{n-1}y^{n-1}+...+a_0(x)=0
\enn

since those of order $n>5$ may not have an explicit representation.

\subsection{Polynomials}
Polynomials are defined as a class of functions of the form
\bnn y = a_0 + a_1 \, x + a_2 \, x^2 + \dots + a_N \, x^N = \sum_{n=0}^N a_n \, x^n \enn
where the function $y$ is said to be a polynomial of order $N$.  \svs

{\bf Example:}
\bnn y = \underbrace{2}_{a_2} x^2 + \underbrace{8}_{a_1} x + \underbrace{4}_{a_0} \enn

{\bf Goal:} To achieve a qualitative understanding of a given function without computing each value.

{\bf Approach:}
\bnn y = \sum_{n=0}^N a_n \, x^n  \qquad \mbox{where we assume} \;\; a_N>0 \enn

\begin{description}
\item[{\bf Step 1:}] If $N$ is even, then $x \rightarrow \pm\infty : y \rightarrow \pm \infty \, ; \quad$
if $N$ is odd, then $x \rightarrow \pm \infty : y \rightarrow \mp \infty$. Note: If $a_N<0$,
the behavior is the opposite.
\item[{\bf Step 2: (Fundamental theorem of algebra)}] A polynomial of order $N$ has $N$ roots which are the solutions of $f(x)=0$.

Set $y=0: f(x) = x^N + a_{N-1}\, x^{N-1} + \dots + a_1 \,x + a_0 = 0 $
\bnn \begin{array}{ccl} \svs
y = 17x + 4 \quad & y=0: x=-\frac{4}{17} & \quad \rightarrow  \quad \mbox{1 root} \\
y = 2x^2 + 8x + 4 \quad & y=0: x=\frac{1}{4}(-8 \pm \sqrt{8^2-4\cdot2\cdot4})=\frac{1}{4}(-8 \pm
\sqrt{32)}=-2 \pm \sqrt{2} & \quad \rightarrow \quad  \mbox{2 roots}
\end{array} \enn
\begin{center}
    $\Rightarrow$ the roots are the locations where $f(x)$ crosses the $x$-axis.
\end{center}
\end{description}

{\bf Note:} Analytic formulas to find the roots of polynomials are known until 4th degree.

{\bf Examples:} Construct graph of $y=f(x)$ \vs
\begin{figure}[!h]
\centering \subfigure[Step 1: $y=x^2+3x+2$] {\epsfxsize=7cm \epsfbox{matlab/fig8a.eps}}
\hspace*{0.5cm}
\subfigure[Step 2:  $y=x^2+3x+2=0$] {\epsfxsize=7cm \epsfbox{matlab/fig8b.eps}} \svs 
\caption{Graphical construction (the roots are $x=-1$ and $x=-2$)} \label{fig8}
\end{figure}

\vs\vs\begin{figure}[!h]
\centering \subfigure[Step 1: $y=x^3+2x^2-3x$] {\epsfxsize=7cm \epsfbox{matlab/fig9a.eps}}
\hspace*{0.5cm}%
\subfigure[Step 2: $y=x^3+2x^2-3x=0$] {\epsfxsize=7cm \epsfbox{matlab/fig9b.eps}} \svs 
\caption{Graphical construction (the roots are $x=-3, x=0$ and $x=1$)} \label{fig9}
\end{figure}

\begin{figure}[!h]
\centering \subfigure[Step 1: $y=x^4+2x^3-3x^2$] {\epsfxsize=7cm \epsfbox{matlab/fig10a.eps}}
\hspace*{0.5cm}%
\subfigure[Step 2: $y=x^4+2x^3-3x^2=0$] {\epsfxsize=7cm \epsfbox{matlab/fig10b.eps}} \svs 
\caption{Graphical construction (the roots are $x=-3, x=0$ and $x=1$)} \label{fig9}
\end{figure} 

\newpage

{\bf Horizontal translation} is a horizontal shift of a function by $x_0$
\bnn y=f(x) \; \rightarrow \; y =f(x-x_0) \enn

{\bf Example:} $\qquad y = x^2$  \; shift by $\; x_0=2$: \; $ y = (x-2)^2 = x^2 - 4x +4$  \vs

\vs {\bf Vertical translation} is a vertical shift of a function by $y_0$
\bnn y=f(x) \; \rightarrow \; y =f(x)+y_0 \enn

{\bf Example:} $\qquad y=x^2$ \; shift by $\; y_0=2$: \;  $ y=x^2+2 $ \vs

\vs\begin{figure}[!h]
\centering \subfigure[Horizontal shift by $x_0=2: y=(x-2)^2$]{\epsfxsize=7cm \epsfbox{matlab/fig11a.eps}} 
\hspace*{0.5cm}
\subfigure[Vertical shift by $y_0=2: y = x^2+2$] {\epsfxsize=7cm \epsfbox{matlab/fig11b.eps}} \svs 
\caption{Vertical and horizontal shift of $y=x^2$} \label{fig11}
\end{figure}

\subsection{Rational functions}

Given two polynomial functions $D(x)$ and $d(x)$ a rational function has the form:
\bnn f(x)=\frac{D(x)}{d(x)} \enn

Working with these functions can be cumbersome if the polynomials are complicated. It is
useful to learn how to factorize the polynomials, because it will allow us to find equivalent
(simplified) functions. 

{\bf 1. Binomial expressions.} Second order polynomials can be factorized in the product of two first order polynomials, i.e. $ax^2+bx+c = (x+u)(x+v)$. Some immediate factorizations are:

\bnn
 	a^2x^2+b^2+2abx=(ax+b)(ax+b)\\
 	a^2x^2+b^2-2abx=(ax-b)(ax-b)\\
    a^2x^2-b^2=(ax+b)(ax-b)
\enn 

Nevertheless, other expressions are not so easy. In general, we know that we are looking
for something like $(x+u)(x+v)$ and hence, if we have for instance

\bnn x^2+bx+c = (x+u)(x+v)= x^2 + (u+v)x + uv, \enn

we can see that $u+v=b$ and $uv=c$, which are two equations with two unknowns that we can solve. A little
bit more general expression would be:

\bnn ax^2+bx+c = (ax+u)(x+v)= ax^2 + (u+av)x + uv, \enn

leads to $u+av=b$ and $uv=c$. 

{\bf Example:}

\bnn
\frac{x^4+8x^2+7}{3x^5-3x} = \frac{(x^2+7)(x^2+1)}{3x(x^2+1)(x^2-1)} = \frac{x^2+7}{3x(x^2-1)}
\enn

%If we substitute $u=c/v$ in the first equation
%we get $c+av^2=bv \quad\rightarrow \quad v=\frac{-b\pm\sqrt{b^2-4ac}}{2a}$ and only one of the
%solutions will be compatible with the second equation $u=c/v$. However, the factorization
%of $c$ in prime numbers typically brings an immediate answer.

{\bf 2. Degree of $D(x)$ is larger than $d(x)$. } In this situation we can simply divide the numerator
by the denominator! Calling $q(x)$ to the quotient and $r(x)$ to the remainder of the division, we get:

\bnn
	\frac{D(x)}{d(x)}=q(x)+\frac{r(x)}{d(x)}.
\enn

{\bf Example:}

\bnn
	\frac{3x^3-2x^2+4x-3}{x^2+3x+3}=(3x-11)+\frac{28x+30}{x^2+3x+3},
\enn

{\bf 3. Degree of $D(x)$ is smaller than $d(x)$. } To solve this case we proceed in three steps that we illustrate with one example:

{\em Step 1 --} Factorize the denominator.
\bnn
	\frac{x}{x^2-3x+2}=\frac{x}{(x-1)(x-2)}
\enn
{\em Step 2 --} Rewrite the function as a sum of rational functions whose numerators are polynomials of degree zero, sum up the functions and group the terms by their degree on $x$.
\bnn
	\frac{x}{(x-1)(x-2)}=\frac{A}{(x-1)}+\frac{B}{(x-2)}=\\
	=\frac{A(x-2)+B(x-1)}{(x-1)(x-2)}=\frac{(A+B)x-(2A+B)}{(x-1)(x-2)}
\enn
{\em Step 3 --} Identify the terms in the numerator of the last expression with those of the original function, and find the coefficients $A$, $B$, etc.
\bnn
	(A+B)x-(2A+B) = x.
\enn

Since the coefficient of $x$ in the numerator of the original function is 1 we get $A+B=1$ and, since the independent term is zero, we get $2A+B=0$. Considering both equations for the variables $A$ and $B$ we conclude that $A=-1$ and $B=2$, and the final factorization is:
\bnn
	\frac{x}{(x-1)(x-2)}=\frac{-1}{(x-1)}+\frac{2}{(x-2)}
\enn

{\bf Example:} It may happen that the multiplicity of any of the factors in the denominator is larger than one, in which case the decomposition at Step 2 becomes:

\bnn
	\frac{x}{(x-1)^3}=\frac{A}{(x-1)}+\frac{B}{(x-1)^2}+\frac{C}{(x-1)^3}
\enn

and we proceed similarly

\bnn
	\frac{x}{(x-1)^3}=\frac{A(x-1)^2+B(x-1)+C}{(x-1)^3}=\frac{Ax^2+(B-2A)x+(A+C-B)}{(x-1)^3}
\enn

finding that $A=0$, $B=1$ and $C=1$.

\subsection{Trigonometric Functions}
Trigonometric functions are a class of periodic functions such as
\bnn y = f(x) = A \sin(kx + \phi) \quad \mbox{and} \quad
     y = f(x) = A \cos(kx + \phi) \enn

The constant parameters are the amplitude
$A$, the frequency $k$ and the phase (angle) $\phi$. The period is
defined as $\lambda=2\pi/k$ and the trigonometric functions fulfill
the relation $f(x+\lambda)=f(x)$.
\small  \vspace*{-5mm} \begin{center} \begin{tabular}{|c|c|c|c|c|c|c|c|c|} \hline
\multicolumn{9}{|c|} {Special values of trigonometric functions (midnight stuff)\rule{0pt}{4mm}} \\
\hline \hline
$x$ & $\;0\;$ & $\pi/6=30^{\circ}$ & $\pi/4=45^{\circ}$ & $\pi/3=60^{\circ}$ & $\;\pi/2=90^{\circ}\;$ & $\;\pi=180^{\circ}\;$
 & $\;3\pi/2=270^{\circ}\;$ & $\;2\pi=360^{\circ}\; $  \rule{0pt}{4mm} \\
\hline
$\;\sin x\;$ & $0$ & $1/2$ & $\sqrt{2}/2$ & $\sqrt{3}/2$ & $1$ & $0$ & $-1$ & $0$  \rule{0pt}{4mm} \\
\hline
$\cos  x$    & $1$ & $\sqrt{3}/2$ & $\sqrt{2}/2$ & $1/2$ & $0$ & $-1$ & $0$ & $1$ \rule{0pt}{4mm} \\
\hline
\end{tabular} \end{center} \normalsize

Horizontal translation (shift) by means of $\phi$:
\bnn
\sin(x+\frac{\pi}{2})=\cos{x} \qquad \cos(x+\frac{\pi}{2})=-\sin{x} \qquad
\sin(x-\frac{\pi}{2})=-\cos{x} \qquad \cos(x-\frac{\pi}{2})=\sin{x}
\enn

Useful relations between $\cos{x}$ and $\sin{x}$:
\bnn \cos^2{x}+\sin^2{x}=1 \enn
\bnn \sin(x\pm y)=\sin{x}\cos{y} \pm \cos{x}\sin{y} \qquad
\cos(x\pm y) = \cos{x}\cos{y} \mp \sin{x}\sin{y} \enn
\bnn \sin{2x} = 2\sin{x}\cos{x} \qquad
\cos{2x} = \cos^2{x} - \sin^2{x} \enn

Other trigonometric functions:
\bnn \tan{x} = \frac{\sin{x}}{\cos{x}} \qquad\qquad \cot{x} = \frac{\cos{x}}{\sin{x}} \enn

\begin{figure}[!h]
\centering 
\subfigure[The functions  $y\!=\!\sin{x}$ and \mbox{$y\!=\!\cos{x}$}] {\epsfxsize=7cm \epsfbox{matlab/fig12a.eps}}
\hspace*{0.5cm}
\subfigure[The functions $y\!=\!\tan{x}$ and $y\!=\!\cot{x}$. Note that these are $\pi$-periodic]
           {\epsfxsize=7cm \epsfbox{matlab/fig12b.eps}} \svs
\caption{The most common trigonometric functions} \label{fig12}
\end{figure}


\subsection{Exponential Functions}

Exponential functions are functions most commonly used in the form
\bnn y = A e^{kx} = A \exp{k x} \enn
with the constant parameters: $A$ amplitude, $k$ growth rate if $k>0$ and the damping or fall 
off, if $k<0$, and $e$ Euler number: 2.718....

Note that $e^0=1\;$ and $\;e^{-x}=\frac{1}{e^x}$. \svs

\begin{figure}[!h]
    \centerline{\epsfxsize=10cm \epsfbox{matlab/fig13.eps}}
    \caption{Exponential functions} \label{fig13}
\end{figure} \vs


\subsection{Hyperbolic Functions}

Hyperbolic functions are of the form
\bnn y = f(x) = \cosh{x} = \frac{1}{2}(e^x + e^{-x}) \qquad\mbox{hyperbolic cosine} \enn
and
\bnn y = f(x) = \sinh{x} = \frac{1}{2}(e^x - e^{-x}) \qquad\mbox{hyperbolic sine} \enn

They have similar properties as the trigonometric functions such as a representation by
exponentials (as we shall see later), and their derivatives convert into each other. 
But the hyperbolic functions are {\em not} periodic. \vs

Other hyperbolic functions:
\bnn
y=\tanh{x}=\frac{\sinh{x}}{\cosh{x}}=\frac{e^{x}-e^{-x}}{e^{x}+e^{-x}} \qquad
y=\coth{x}=\frac{\cosh{x}}{\sinh{x}}=\frac{e^{x}+e^{-x}}{e^{x}-e^{-x}}
\enn

\begin{figure}[!h]
    \centering
    \subfigure[$y=\sinh{x}\;$ and $\;y=x+\frac{1}{3!}\,x^3$]{\epsfxsize=7cm \epsfbox{matlab/fig14b.eps}}
    \hspace*{0.5cm}
    \subfigure[$y=\cosh{x}\;$ and $\;y=1+\frac{1}{2!}\,x^2$]{\epsfxsize=7cm \epsfbox{matlab/fig14a.eps}} \svs
    \caption{Hyperbolic functions} \label{fig14}
\end{figure} \vs



\subsection{Basic Inverse Functions}
\subsubsection{Logarithms}
The logarithms are the inverse of the exponential functions:
\bnn y=a^x \quad \leftrightarrow \quad x = \log_a y  \qquad\mbox{where} \quad 0<y<\infty \enn

{\bf Special cases:}
\begin{eqnarray*}
\begin{array}{lcccccc}
a=e: \qquad & y & = & \log_e x & = & \ln x & \qquad\mbox{natural logarithm}\\
a=10: \qquad & y & = & \log_{10} x & = & \mbox{lg} x & \qquad\mbox{decimal logarithm}\\
a=2: \qquad  & y & = & \log_2 x & = & \mbox{ld} x & \qquad\mbox{dual logarithm}
\end{array}
\end{eqnarray*}

\begin{figure}[!h]
    \centerline{\epsfxsize=10cm  \epsfbox{matlab/fig15.eps}} \svs
   \caption{The logarithmic functions $y=\ln(x)$, $y=\mbox{lg}(x)$, $y=\mbox{ld}(x)$, and $y=-\ln(x)$.} \label{fig15}
\end{figure} \vs

{\bf Remark:} The most commonly used logarithm is $\ln x$, but there are certain applications for other 
logarithms as well. For instance, the decimal logarithm can be used to find the number of digits in 
a decimal number 
($\mbox{lg} 4821=3.683 \rightarrow$  taking the whole number in front of the decimal point and 
adding 1 gives the number of digits, 4). Similarly, the dual logarithm can be used to find the number
of bits or binary digits that are necessary to represent a number $n$ in binary format, i.e. as 
zeros and ones.  

{\bf Rules and tricks for dealing with logarithms:}
\bnn \log_a x^n = n \log_a x \enn
\bnn \log_a x_1 \, x_2 = \log_a x_1 + \log_a x_2  \qquad  \log_a \frac{x_1}{x_2} = \log_a x_1 - \log_a x_2  \enn
\bnn \log_a x = \frac{\log_b x}{\log_b a} \enn

{\bf Note:} The last expression tells us that changing the base simply changes the values of the function by a constant. Every logarithm can be expressed in terms of the natural logarithm, and every exponential 
function can be written in terms of the basis $e$

A particularly useful relation is the following:

\bnn    a^x = e^{\ln a^x} = e^{x\,\ln a} \quad \mbox{with} \quad a>0 \enn 


\subsubsection{Other inverse functions}
\vspace*{-0.9cm} \begin{eqnarray*} \begin{array}{llll}
y=\sin x & \rightarrow & x = \arcsin y & \mbox{arc sine} \\
y=\cos x & \rightarrow & x = \arccos y & \mbox{arc cosine} \\
y=\tan x & \rightarrow & x = \arctan y & \mbox{arc tangent} \\
y=\cot x & \rightarrow & x = \mbox{arccot } y & \mbox{arc cotangent} \\
y=\sinh x & \rightarrow & x = \mbox{arcsinh } y & \\
& & x=\ln(y+\sqrt{y^2+1})
& \mbox{area sine hyperbolic} \\
y=\cosh x & \rightarrow & x = \mbox{arccosh } y & \\
& & x = \ln(y+\sqrt{y^2-1})
& \mbox{area cosine hyperbolic}
\end{array} \end{eqnarray*} \svs

\begin{figure}[!h]
    \centering
    \subfigure[Arc sine, and arc cosine]{\epsfxsize=7cm \epsfbox{matlab/fig15_1a.eps}}
    \hspace*{0.5cm}
    \subfigure[Arc tangent, and arc cotangent]{\epsfxsize=7cm \epsfbox{matlab/fig15_1b.eps}} \svs
    \caption{The inverse of the trigonometric functions}
\end{figure}

\subsection{Elementary Combinations of Functions}
\subsubsection{Superposition}
Two functions are superimposed on each other by adding their values for the same $x$.
\bnn y = f_1(x) + f_2(x) \enn
\vspace*{-.5cm} \begin{figure}[!h]
    \centering
    \subfigure[Superposition of $f_1(x)\!=\!-x \mbox{ and } f_2(x)\!=\!1$]{\epsfxsize=7cm \epsfbox{matlab/fig16a.eps}}
    \hspace{0.5cm}
    \subfigure[Superposition of $f_1(x)\!=\!-2x \mbox{ and } f_2(x)\!=\!x^3$]{\epsfxsize=7cm \epsfbox{matlab/fig16b.eps}} \svs
    \caption{Superposition of lines and functions} \label{fig17}
\end{figure}

\subsubsection{Modulation}
A function is modulated by another function by multiplying their values for the same $x$. 
\bnn y = f_1(x) \, f_2(x) \enn

\vs\begin{figure}[!h]
    \centering
    \subfigure[$e^{-x}$ is the envelope function of $\cos 10x$] {\epsfxsize=7cm \epsfbox{matlab/fig17a.eps}}
    \hspace{0.5cm}
    \subfigure[$\cos 10x$ is modulated with $\sin x$]{\epsfxsize=7cm \epsfbox{matlab/fig17b.eps}}  \svs
    \caption{Modulation of functions} \label{fig19}
\end{figure}

\newpage