%% LyX 2.1.4 created this file.  For more info, see http://www.lyx.org/.
%% Do not edit unless you really know what you are doing.
\documentclass[english]{article}
\usepackage[T1]{fontenc}
\usepackage[latin9]{inputenc}
\usepackage{geometry}
\geometry{verbose,tmargin=2cm,bmargin=2.5cm,lmargin=2.3cm,rmargin=2cm}
\usepackage{amstext}
\usepackage{amssymb}

\makeatletter
%%%%%%%%%%%%%%%%%%%%%%%%%%%%%% Textclass specific LaTeX commands.
\newcommand{\lyxaddress}[1]{
\par {\raggedright #1
\vspace{1.4em}
\noindent\par}
}

\makeatother

\usepackage{babel}
\begin{document}

\title{SURVIVAL TOOLBOX - Maths for Biology}


%\date{Course 2017-18}


\author{Computational Methods in Ecology and Evolution \\ Imperial College
London \\ Silwood Park}

\maketitle

\lyxaddress{Lecturer: Alberto Pascual-Garc�a. \\ }


\lyxaddress{Correspondence: alberto.pascual.garcia@gmail.com}


\subsection*{Trigonometric relationships}
\begin{enumerate}
\item $\sin(x+\frac{\pi}{2})=\cos x$; $\cos(x+\frac{\pi}{2})=-\sin x$;
$\sin(x-\frac{\pi}{2})=-\cos x$; $\cos(x-\frac{\pi}{2})=\sin x$; 
\item 
\[
\sin^{2}x+\cos^{2}x=1.
\]

\item 
\[
\sin(x+y)=\sin x\cos y+\sin y\cos x
\]
(Particularly important is $\sin2x=2\sin x\cos x$).
\item 
\[
\cos(x+y)=\cos x\cos y-\sin x\sin y
\]
(Again, when $x=y$ we obtain $\cos(2x)=\cos^{2}x-\sin^{2}x$).
\item 
\[
\tan(x+y)=\frac{\tan x+\tan y}{\text{1-\ensuremath{\tan}x\ensuremath{\tan}y}}
\]
(For $x=y$ leads to $\tan2x=\frac{2\tan x}{1-\tan^{2}x}$).
\item 
\[
\sin x+\sin y=2\sin\frac{x+y}{2}\cos\frac{x-y}{2}.
\]

\item 
\[
\cos x+\cos y=2\cos\frac{x+y}{2}\cos\frac{x-y}{2}.
\]

\item 
\[
\cos x-\cos y=-2\sin\frac{x+y}{2}\sin\frac{x-y}{2}.
\]

\item 
\[
\sin^{2}x=\frac{1-\cos2x}{2}
\]
 and 
\[
\cos^{2}x=\frac{1+\cos2x}{2}.
\]

\item 
\[
\sin^{2}x=\frac{\tan^{2}x}{1+\tan^{2}x}
\]
 and 
\[
\cos^{2}x=\frac{1}{\text{1+\ensuremath{\tan}}^{2}x}.
\]

\item 
\[
\sin x=\frac{2\tan\left(x/2\right)}{1+\tan^{2}(x/2)}
\]
 and 
\[
\cos x=\frac{1-\tan(x/2)}{\text{1+\ensuremath{\tan}}^{2}(x/2)}.
\]

\end{enumerate}

\subsection*{Logarithmic relationships}
\begin{enumerate}
\item For all $x,y\in\mathbb{\mathbb{R}}$ 
\[
\log_{2}xy=\log_{a}x+\log_{a}y
\]
 and 
\[
\log_{a}x^{y}=y\log_{a}x.
\]

\item For all $x,y\in\mathbb{\mathbb{R}}$ with $y\neq0$
\[
\log_{a}\frac{x}{y}=\log_{a}x-\log_{a}y.
\]

\item For all $x,y\in\mathbb{\mathbb{R}}$ with $y\neq0$
\[
\log_{b}x=\frac{\log_{a}x}{\log_{a}b}
\]
 and 
\[
a^{x}=b^{x\log_{b}a}.
\]
 These relationships are particularly important for the neperian logarithm:
\item 
\[
\log_{b}x=\frac{\ln x}{\ln b}
\]
 and 
\[
a^{x}=e^{x\ln a}.
\]

\end{enumerate}

\subsection*{Infinitesimal equivalences}

The following are equivalent infinitesimal when $x\rightarrow0:$
\begin{enumerate}
\item 
\[
\sin x\approx x\approx\arcsin x.
\]

\item 
\[
\tan x\approx x\approx\arctan x.
\]

\item 
\[
1-\cos x\approx\frac{x^{2}}{2}.
\]

\item 
\[
\log(1+x)\approx x.
\]

\item 
\[
a^{x}-1\approx x\log a
\]
 for $a>0,a\neq1.$\end{enumerate}

\end{document}
